\section{Contact Force Estimation}
	\label{sec:contact_estimation}
	
	In most robotics applications, robots come into contact with the environment just with
	their end-effectors.
	Nevertheless, robots could benefit from making contacts with other parts of their bodies,
	in the same way humans do.
	This lack is mainly due to the fact that nowadays most robots do not have an artificial skin
	that allows them to detect and localize contacts.
	Either joint torque sensors or 6 axis Force/Torque (F/T) sensors are usually used to provide
	contact	feed-back to robots.
	However, with these sensors it is not possible to retrieve the exact contact location,
	unless we make strong assumptions about the contact
	(e.g. zero moment applied, known force direction).
	Moreover, the wrench (i.e. force and moment) that is applied at the contact point cannot be
	measured, unless the contact location is somewhat known \cite{DelPrete_PhDThesis2013}.
	
	Within this context, Sentis et al. \cite{Sentis_IROS2009} presented a theoretical framework
	to model and control robots that are subject to multiple contacts, but the authors didn't
	discuss how to localize the contacts and estimate the	contact forces / pressures.
	On the other hand, Park and Khatib \cite{Park_ICRA2005} presented a compliant motion control
	framework for multiple contacts and they tested it with a PUMA560 manipulator.
	In the experiments the geometry and the stiffness of the environment were assumed to be known
	a priory in order to compute the contact points and the contact forces.
	
	A probabilistic approach was later proposed by Petrovskaya et al. \cite{Petrovskaya_ICRA2007},
	in which the authors used	an active sensing strategy to estimate at the same time the shape of
	the robot and the contact point.
	Nevertheless, they tested the method only with extremely simple geometries of robot and
	environment (the environment was a point and the robot link was a line) and the authors said
	that,	for more complex geometries, more sophisticated active exploration strategies would likely
	be needed \cite{DelPrete_PhDThesis2013}.
	
	When both tactile sensors and F/T sensors are available, it is possible to estimate contact
	locations and	contact wrenches, and so implement reliable contact force control.
	This was the approach taken by Del Prete \cite{DelPrete_PhDThesis2013} who implemented the
	corresponding method on the iCub humanoid	robot \cite{Metta_PerMIS2008}, an open robotic
	platform designed for studying embodied cognition featuring 6-axis F/T sensors and a distributed
	sensorized skin.
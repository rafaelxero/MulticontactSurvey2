\section{Areas of inspiration}
	\label{sec:inspiration}
	
	Force distribution due to multi-contact interactions in \emph{multi-legged walking robots} has been
	addressed since the early 80's.
	Then, research began to focus on modeling multi-grasp behaviors in \emph{dexterous mechanical hands}
	and the associated internal forces acting between \emph{multiple cooperating manipulators};
	that is, to a number of simple closed	multiple-chain robotic systems \cite{Orin_AdvRobotics1989}
	\cite{Nahon_TransRoboticsAuto1992} \cite{Chen_MIRC1999} \cite{Sentis_TransRobotics2010}.
	These fields provided the inspiration and tools that were later used to implement	multi-contact force
	control in humanoid robots.
	
	\subsection{Multi-legged walking robots}
		\label{sub:walking_robots}
		
		Some of the earliest works in the area of multilegged walking robots was the one of Orin and Oh
		\cite{Orin_DSMC1981} in 1981, where a solution that	optimized a weighted combination of energy
		consumption and load balance was used on simple closed-chain mechanisms in which a single member,
		called the reference member, is supported by several chains.
		While the model included all three types of robotics systems of interest, it was applied to	multilegged
		vehicles only, specifically to drive a hexapod locomotion vehicle.
		Hard point contact with friction was used to model the foot / support-surface interaction.
		Joint actuator limits and leg dynamic effects were also included in the formulation.
		However, the method used to include these was computationally inefficient.
		The above statement was done by Orin and Chen \cite{Orin_AdvRobotics1989}, which in turn presented a
		computationally efficient formulation of the force distribution problem which included the dynamic
		effects of the chains and physical limits on their actuators, as well as used a contact
		modeling relatively general and capable of handling hard point contact, soft finger contact,
		or rigid contact with an irregular-shaped	object or with uneven terrain.
		
		Later, at the beginning of the 90's, Klein and Kittivatcharapong \cite{Klein_RoboticsAuto1990}
		proposed to solve the force distribution problem for the limbs of a legged vehicle with friction
		cone constraints, while Kumar and Waldron \cite{Kumar_MechDesign1990} addressed the problem of the
		appropriate distribution of forces between the legs of a legged locomotion system for walking
		on uneven terrain.
		Also, Gardner and Srinivasan \cite{Gardner_DSMC1990} solved the force distribution
		problem in a closed form for a walking machine such that it would be computationally simpler,
		and after Gardner \cite{Gardner_DSMC1991} extended the previous work to allow for
		arbitrarily oriented surface normals at the point of contact between the feet and the ground.
		
		At the end of the 90's Liu and Wen \cite{LiuH_RoboticSystems1997} focused on preventing leg
		slippage by means of an efficient approach to optimize the foot force distribution on a quadruped
		walking vehicle.
		Later, Chen et al. \cite{Chen_MIRC1999} developed a real-time force control for a
		quadruped robot by transforming the friction constraints from nonlinear inequalities into a
		combination of linear	equalities and linear inequalities reducing the problem size, and
		Hung et al. \cite{Hung_SysManCyb2000} presented a systematic formulation of the force
		distribution equations for a general tree-structured robot mechanism.
		
	\subsection{Multiple cooperating manipulators}
		\label{sub:cooperating_manipulators}
	
		In the field of cooperation (or coordination) of multiple manipulators some representative works
		that date from the 90's also made use of optimal force distribution.
		For example, Shin and Chung \cite{Shin_IROS1991} proposed a method called weak point minimization
		applicable to weakly connected assembly parts and weak joints in cooperating multiple robots.
		Choi et al. \cite{Choi_ICRA1992} found an optimal load distribution for two cooperating robots by
		utilizing a force ellipsoid, a concept that was also used together with the manipulability ellipsoid
		by them later \cite{Choi_Robotica1993}.
		Also, Kown and Lee \cite{Kwon_SICE1996} proposed a compact dual method for multiple cooperating
		robots, and after they \cite{Kwon_IntellRobotSys1998} used quadratic constraints to reduce their
		number and improve efficiency.
		Finally, Featherstone et al. \cite{Featherstone_ICRA1999} presented a general first-order kinematic
		model of	frictionless rigid-body contact for use in hybrid force / motion control that made it
		possible to include multiple points of contact.
		
	\subsection{Dexterous mechanical hands}
		\label{sub:mechanical_hands}
		
		As for grasping with dexterous mechanical hands, there are also some representative works published
		also around the 90's.
		Cheng and Orin \cite{Cheng_ICRA1989} \cite{Cheng_TransRoboticsAuto1990} used the duality theory of
		linear programming to obtain the general solution of linear equality constraints and applied it to
		a grasping system, and later they \cite{Cheng_TransSysManCyb1991a} \cite{Cheng_TransSysManCyb1991b}
		proposed a computationally efficient formulation applicable to multiple-chain robotic systems.
		Mukherjee and Waldron \cite{Mukherjee_MechDesign1992} proposed to minimize the maximum value of
		the friction angle at the points of contact of a three-fingered hand.
		Later, Chen \cite{Chen_ICRA1995} analyzed the mechanics of grasping general solid objects under a
		frictional point contact model and proposed to balance disturbances by means of a particular set
		of internal forces, while Cheng \cite{Cheng_SysManCyb1997} proposed an efficient method for obtaining
		the general solution for the force balance equations with hard point contacts.
		Also, Liu \cite{LiuY_TransRoboticsAuto1999} formalized a qualitative test of 3D frictional
		form-closure grasps, and Al-Gallaf \cite{AlGallaf_RobAutoSys2006} presented a novel neural
		network for dexterous hand-grasping inverse kinematics mapping used in force optimization.
		
	\subsection{Non-specialized field of application}
		\label{sub:non_specialized}
		
		Some works were not mainly focused on one applicable field of the optimal force distribution,
		but on its application to closed kinematic chains, comprising the above stated fields.
		For example, Nahon and Angeles \cite{Nahon_TransRoboticsAuto1992} focused on solving the
		optimization problem with equality and inequality constraints in order to achieve real-time
		control of cooperating manipulators, mechanical hands and walking machines.
		In contrast, Sreenivasan et al. \cite{Sreenivasan_MechDesign1996} used the redundancy in actuation
		of multifingered hands and walking vehicles to optimize the force distribution, but the developed
		algorithm was not suited for real-time implementation, just for offline planning.
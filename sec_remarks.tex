\section{Remarks}
	\label{sec:remarks}
	
	Many control frameworks have then been formulated in the last decade, some of them clearly
	representative, e.g. Inverse-Kinematics Force Control (IKFC) \cite{Fujimoto_ICRA1996},
	Virtual Model Control (VMC) \cite{Pratt_IJRR2001}, Operational Space Based Control
	(OSBC) \cite{Sentis_MotionPlan2010}, Passivity-Based Balance Control (PBBC)
	\cite{Hyon_TransRobotics2007}, Floating Body Inverse Dynamics (FBID) \cite{Mistry_ICRA2010},
	Dynamic Balance Force Control (DBFC) \cite{Stephens_IROS2010}, Model Preview Control (MPC)
	\cite{Nagasaka_RobotSymp2012}, Task Space Inverse Dynamics (TSID) \cite{DelPrete_PhDThesis2013},
	Momentum-Based Control (MBC) \cite{Hopkins_IJHR2016}, among others.
	
	Each of these frameworks present particular features \cite{DelPrete_PhDThesis2013}, such as:
	specification of inequality constraints \cite{Mansard_TransRobotics2009} \cite{Saab_ICRA2011},
	control of underactuated systems \cite{DeLasa_IROS2009} \cite{Mistry_RSS2011}, low computational
	cost \cite{Escande_ICRA2010} \cite{Mansard_ICRA2012}, etc., but in general they pursue one
	common objective: to create complex behaviors, for which humanoid robots need to simultaneously
	accomplish multiple control objectives.
	For instance, locomotion, manipulation, balance, and posture stance need to be simultaneously
	controlled \cite{Sentis_PhDThesis2007}.
	
	It is possible to identify two different approaches to multi-task management among the described
	frameworks \cite{DelPrete_PhDThesis2013}:
	%
	\begin{enumerate}
		\item The \emph{error minimization} approach.
					This one minimizes the error of each task, under the constraint of not conflicting
					with higher priority tasks.
					Unfortunately, it suffers from algorithmic singularities, namely singularities that
					are due to conflicts between tasks.
					
		\item The \emph{null-space projection} approach.
					This one is somewhat simpler as it only projects tasks into the null-space of the
					constraints.
					However, it has a drawback, as it does not ensure that a	task is achieved,
					even if it is not in conflict with any task with higher priority.
		
	\end{enumerate}
		
	Despite the drawbacks of the null-space projection approach sometimes it has been preferred
	over the error minimization approach \cite{DelPrete_PhDThesis2013}, as it was done by
	Khatib	\cite{Khatib_IJRR1995}, Chiaverini \cite{Chiaverini_RoboticsAuto1997},
	Peters et al. \cite{Peters_AutoRobots2008}, or Mistry and Righetti \cite{Mistry_RSS2011}. 
	This is because
	%
	\begin{inparaenum}
		\item it is older and hence more popular in the robotics community,
		\item it is simpler to understand and to implement, and
		\item it does not suffer from algorithmic singularities.
	\end{inparaenum}
	%
	However, it is not optimal \cite{DelPrete_PhDThesis2013}.
		
	On the other hand, if the robot has to perform two or more tasks with different priorities,
	then the error minimization approach has to be taken to get optimal results.
	Many frameworks for multi-task control that follow the principles of this approach have been
	presented.
	Some examples are the works of Siciliano and Slotine \cite{Siciliano_AdvRobotics1991},
	Jeong and Chang \cite{Jeong_IROS2009} or Saab et al. \cite{Saab_ICRA2011}.
	
	Even the work of Khatib	\cite{Khatib_IJRR1995}, which in its original form used the null-space
	projection approach, has been extended later to the error minimization approach in the works
	of Khatib et al. \cite{Khatib_IJHR2004}, or Sentis and Khatib \cite{Sentis_ICRA2006}, as it
	was interpreted by Del Prete \cite{DelPrete_PhDThesis2013}.
		
	In a general walking robot, the number of constraints is larger than that of a finger system.
	Therefore, it may be difficult to implement this method in real-time without reducing the
	problem size \cite{Kwon_IntellRobotSys1998}.
	This reduction of the problem size represents an approach that can improve the performance
	of this method, as proposed by Chen et al. \cite{Chen_MIRC1999}.
	
	The use of Quadratic Programming (QP) in online control algorithms has recently seen a sudden
	rise in popularity,	at least in part due to the availability of fast and reliable QP solvers
	and more powerful CPUs \cite{Koolen_IJHR2016}.
	An early example of such QP-based control schemes is the work of Kudoh et al. \cite{Kudoh_IROS2002}.
	More recent approaches include the work of Stephens and Atkeson \cite{Stephens_IROS2010},
	Saab et al. \cite{Saab_TransRobotics2013}, Kuindersma et al. \cite{Kuindersma_ICRA2014},
	Herzog et al. \cite{Herzog_IROS2014}, and Feng et al. \cite{Feng_Humanoids2014}
	
	Several researchers have proposed convex optimization techniques to solve the inverse dynamics and
	whole body control problem subject to multi-contact constraints, e.g. De Lasa and Hertzmann
	\cite{DeLasa_IROS2009}, Lee and Goswami \cite{LeeS_IROS2010}, Koolen et al. \cite{Koolen_Humanoids2013},
	Wensing and Orin \cite{Wensing_ICRA2013}, Feng et al. \cite{Feng_Humanoids2014}, Saab et al.
	\cite{Saab_TransRobotics2013} and Kuindersma et al. \cite{Kuindersma_ICRA2014}
	These approaches compute joint torque set-points that minimize tracking errors for multiple motion
	tasks including momentum rates of change, end-effector accelerations, and joint accelerations relating
	to whole-body motions.
	In general, these formulations can serve as the basis for any locomotion, manipulation, or generic
	multi-contact behavior \cite{Hopkins_ICRA2015}.
	With respect to this multi-contact behavior many works in the literature have only focused on
	controlling forces exerted at the	end-effectors, namely the hands for manipulation tasks and
	the feet for walking.
	Restricting contacts to end-effectors is a quite strong assumption, since
	%
	\begin{inparaenum}
		\item uncertainties in the environment may result in unplanned contacts at other body parts,
					and
		\item contacts at other body parts may be necessary to perform certain tasks
					\cite{DelPrete_PhDThesis2013}.
	\end{inparaenum}
	
	Some works do not make any assumption about the contact location, trying to control the joint
	torques	rather than the contact forces.
	This approach ensures bounded contact forces, but do not allow a real contact force control
	\cite{DelPrete_PhDThesis2013}.
	
	Various theoretical frameworks for modeling and control of robots that are subject to multiple
	contacts have been proposed (e.g. Park and Khatib \cite{Park_ICRA2006}, Sentis et al.
	\cite{Sentis_IROS2009} and Righetti et al. \cite{Righetti_ICRA2011}), but they have not
	gone beyond end-effector contacts when tested	on real platforms \cite{DelPrete_PhDThesis2013}.

	One of the first schemes for model-based torque control was due to Khatib
	\cite{Khatib_RoboticsAuto1987}, later extended and applied to humanoid robots.
	Such a framework allowed the specification of (a hierarchy of) multiple motion tasks,
	and could handle the case of multiple non-coplanar ground contacts.
	Then, joint torques were essentially found using unconstrained least squares (pseudo-inverse) techniques
	\cite{DelPrete_PhDThesis2013}.
	Examples of this type of scheme include the work of Hyon et al. \cite{Hyon_TransRobotics2007},
	Righetti et al. \cite{Righetti_CLAWAR2010} and Mistry et al. \cite{Mistry_ICRA2010}
	
	A consequence of this unconstrained optimization is that limitations due to unilateral ground contact
	and friction cannot be taken into account directly \cite{Koolen_IJHR2016}.
	Also, the control laws based on the assumption of no motion along the constrained directions can
	generate constraint forces that are not physically feasible.
	Righetti et al.	\cite{Righetti_Humanoids2011} tried to solve this issue by using the constraint
	redundancy to minimize a quadratic cost in the constraint forces, but still it couldn't guarantee
	their physical consistency \cite{DelPrete_PhDThesis2013}.
	
	A safer approach was to include inequality constraints into the control problem, as done by Saab et al.
	\cite{Saab_ICRA2011} \cite{Saab_IROS2011} or later by Righetti and Schaal	\cite{Righetti_Humanoids2012},
	so that it allowed for unilateral contact constraints \cite{DelPrete_PhDThesis2013}.
	This consideration naturally led to the use of constrained optimization	techniques.
	But, even though the friction cone constraints related to Coulomb friction were second-order cone
	constraints, these second-order cones were often approximated using linear constraints, reducing the
	problem of finding joint torques given motion tasks to a quadratic program (QP) \cite{Koolen_IJHR2016}.
	This is computationally more expensive than a solution based on pseudo-inverses; however, it is
	claimed that the computation time of the controller can be still less than 1 ms, and so perfectly
	suitable for fast torque control loop \cite{DelPrete_PhDThesis2013}.
	
	One limitation of works like the one of Del Prete \cite{DelPrete_PhDThesis2013} is that their control
	framework didn't consider joint limits and motor torque limits.
	These limits can be easily included in the control problem as inequality constraints, using the QP
	solver to compute the control torques \cite{DelPrete_PhDThesis2013}, as it is done by Hopkins et al.
	\cite{Hopkins_IJHR2016} and \cite{Koolen_IJHR2016}.
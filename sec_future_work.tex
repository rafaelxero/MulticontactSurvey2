\section{Future Work}
	\label{sec:future_work}
	
	Some of the recent works on multi-task whole-body multi-contact control have recognized weaknesses
	that have to be overcome by future research efforts.
	Some relevant ones are mentioned in the following:
	%
	\begin{itemize}
					
		\item \emph{Contact behavior}.
					It is advisable for future work to include the implementation of extreme contact behaviors,
					such as those that would exploit point and edge contacts to maneuver with the environment,
					instead of relying on planar surfaces.
					Also, it would be interesting to analyze contact singularities \cite{Sentis_TransRobotics2010}.
					
		\item \emph{Wrench measurement}.
					It is required that the humanoid robots have the capability of simultaneously measuring forces
					and	torques (i.e., wrenches) at any possible contact location.
					This is not possible with conventional torque-controlled manipulators and requires whole-body
					distributed force and tactil sensing \cite{Nori_FrontRobAI2015}.
					Also, it is worth to consider that, for future robot skin, composed of thousands or millions
					of sensors, it would be extremely challenging and time consuming to perform a full spatial
					calibration \cite{Calandra_Humanoids2015}.
		
		\item \emph{State estimation}.
					Future research efforts will have to focus on improved humanoid state estimation, as the
					accuracy of the floating base odometry is believed to be a limiting factor in the current
					implementations	\cite{Hopkins_IJHR2016}.
					
		\item \emph{Robustness and adaptation}.
					Accounting for uncertainties could lead to major improvements in robotics if applied at all
					levels (i.e. planning, control, estimation and identification).
					While robust Model Predictive Control (MPC) is already an active research field, applications
					of robust optimization in robotics are seldom \cite{DelPrete_LAAS2015}.
					Also, if whole-body control approaches become increasingly robust, new opportunities could
					arise for optimization-based locomotion planners that incorporate novel task-space and
					joint-space cost terms and additional constraints \cite{Hopkins_IJHR2016}.
					
		\item \emph{Singularities}.
					Whole-body algorithms that compute joint accelerations based on desired Cartesian accelerations
					require the inversion of a matrix, which can be singular.
					Singularities such as this have been a major issue in robotics and one of the main reasons why
					so many humanoid robots walk with significantly bent knees even though real humans walk with
					fairly straight	knees.
					There have been so many different methods for compensating for singularities, such as using
					damped least squares, but there is still no satisfying solution \cite{Koolen_IJHR2016}.
					
		\item \emph{Failure handling}.
					It is currently interesting to have appropriate methods to handle uncertainty and failure,
					specifically regarding the position and orientation of the assumed contact points, and also
					including the application of adaptive control methods for dynamic step recovery
					\cite{Lengagne_IJRR2013} \cite{Hopkins_IJHR2016}.
					Also, future applications will require automatic generation of stable behaviors.
					In addition, controllers should improve performance and efficiency of behaviors through
					experience \cite{Stephens_PhDThesis2011}.
					
		\item \emph{Complex motions}.
					It is a general goal to gradually increase the complexity of the generated motion:
					to have the robot making dynamic motion while making and breaking contacts with the rigid
					(or non-rigid)	environment, e.g. walking with additional support of the hands
					\cite{Koenemann_IROS2015}.
					It is to be noted that, despite the various studies on angular momentum in humanoid motions,
					the issue of how to set the desired angular momentum for more complex motions has not been
					fully explored \cite{LeeS_AutoRobots2012}.
					
		\item \emph{Computational cost}.
					Several solutions have only been tested in simulation due to the high computational cost of the
					algorithms involved, as well as due to several bottlenecks (e.g. in planning), which makes it
					difficult to allow a real-time execution on the real robot \cite{Kudruss_Humanoids2015}.
					
		\item \emph{Hardware and system integration}.
					Not only will humanoid robot controllers need to improve, but the hardware and system integration
					need to be improved to allow robots capable of real world application \cite{Stephens_PhDThesis2011}.
	
	\end{itemize}
	

	
	

	
	

	

	

	
	

	

	
	
\section{Force Control}
	\label{sec:force_control}
	
	\subsection{Paradigms}
		\label{sub:paradigms}
		
		There are different approaches (or paradigms) to force control that we can find in the
		literature \cite{DelPrete_PhDThesis2013}:
		%
		\begin{itemize}
			
			\item \emph{Explicit force control}.
						This one exploits force feedback to adjust the
						contact force to a desired value \cite{Volpe_TransAutoControl1993}.
			
			\item \emph{Implicit force control}.
						This one, which without any force feedback, regulates the position of the
						contact point, while tuning the joint servo gains so as	to give a particular
						stiffness / admittance to the contact point	\cite{Rocco_Automatica1997}.
			
			\item \emph{Impedance control}.
						Proposed originally by Hogan \cite{Hogan_AmericanCtrlConf1984}.
						This one consists in the control of the motion of the	manipulator and,
						in addition, its dynamic behavior in response to external	forces.
						Based on the observation that all environments can be modeled as
						\emph{admittances} (i.e. physical systems that accept force inputs
						and yield motion outputs), Hogan suggested to control the manipulator as
						an \emph{impedance} (i.e. physical system that accepts motion inputs
						and yield force outputs).
						This guarantees that thew two dynamically interacting systems physically
						complement each other.
						Different from previous approaches, impedance control does not regulate
						motion or force directly, but instead it regulates the ratio of force to
						motion, that is the mechanical impedance \cite{DeShutter_ControlProblemsRobAuto1998}.
			
			\item \emph{Hybrid control}.
						Introduced by Craig and Raibert \cite{Craig_SoftwareApp1979}.
						This one regulates force and position independently, along orthogonal directions.
						Any assembly task involving rigid frictionless contacts defines two sets of
						\emph{natural constraints}: position constraints prevent the robot from moving
						through the environment, whereas force constraints prevent the application of
						forces along the tangent directions.
						These two sets of constraints partition the space into two orthogonal subspaces,
						which are then controlled according to different criteria
						\cite{Khatib_RoboticsAuto1987}.
						
			\item \emph{Hybrid impedance control}.
						Introduced by Anderson and Spong \cite{Anderson_JRobticsAuto1988}.
						This one combines impedance control and hybrid position / force control into one
						strategy, while allowing for more sophisticated impedances.
						
			\item \emph{Parallel control}.
						Proposed by Chiaverini and Sciavicco \cite{Chiaverini_RoboticsAuto1993}.
						It consists on superimposing a proportional-derivative position control on a
						proportional-integral force control.
						The integral action in the force regulator ensures dominance of the force control
						over the position control, which is typically desired for safety reasons.
						
		\end{itemize}
		
	\subsection{Force distribution}
		\label{sub:distribution}
		
		Force control approaches for multiple-chain robotic systems have solved the	force distribution
		problem by means of a variety of methods, once the equations had been fully	derived
		\cite{Orin_AdvRobotics1989}.
		Some of the representative ones, as identified by Orin and Cheng \cite{Orin_AdvRobotics1989} and
		Chen et al. \cite{Chen_MIRC1999},	are:
		
		\begin{enumerate}
		
			\item \emph{Linear-Programming (LP) Method}.
						The problem is formulated as a linear-programming problem, by replacing the
						friction cone with a piecewise linear pyramid so as to express friction constraints by
						linear inequalities.
						However, this method is difficult to be applied to complex systems in real-time, as it leads
						easily to discontinuous solutions even under smooth changes in constraints
						\cite{Nahon_TransRoboticsAuto1992}.
						
			\item \emph{Gradient Projection Method}.
						This one is useful for considering unilateral constraints in a cost function whose
						gradient is projected in the space of free motion as a lowest-priority task
						\cite{Mansard_TransRobotics2009}.
						
			\item \emph{Compact-Dual LP (CDLP) Method}.
						This one results in a smaller problem size by using the compact-dual linear programming,
						but it cannot overcome the discontinuous problem \cite{Nahon_TransRoboticsAuto1992}.
						
			\item \emph{Quadratic Programming (QP) Method}.
						It is superior to LP and CDLP in terms
						of the quality of the solution and also can be implemented in real-time because the time for
						obtaining a solution does not depend on an initial guess~\cite{Chen_MIRC1999}.
						
			\item \emph{Non-Linear Programming (NLP) Method}.
						NLP can be used when the program can be difficult to solve if the local QP is	a poor estimate
						of the true nonlinear program \cite{Posa_WAFR2012}.
						However, it is time consuming and there is no mention about real-time
						implementation~\cite{Sharma_NUiCONE2011}.
						
			\item \emph{Analytical Method}.
						This one is mainly applied to walking robots.
						It is characterized by finding a relation among feet forces in order to prevent legs from
						slipping, and consequently adding equality constraints so that the undetermined force system
						may be transformed into a determined system after combining inverse dynamic equations with the
						equality constraints.
						However, it attempts only to prevent leg slippage, neglecting the other inequality constraints
						\cite{Chen_MIRC1999}.
		
		\end{enumerate}
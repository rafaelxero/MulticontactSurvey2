\section{Multi-task Whole-body Control Frameworks}
	\label{sec:whole_body}
	
	In order to meet the increasing demand of flexibility and versatility, robot manipulators
	with more than six degrees-of-freedom (dof) appeared.
	These manipulators introduced the problem of \emph{redundancy resolution}, namely to select
	a unique solution among the infinite control actions that generate the desired end-effector
	behavior \cite{DelPrete_PhDThesis2013}.
	
	At first, redundancy was exploited to minimize the kinetic/potential energy of the manipulator,
	avoid obstacles in the workspace, avoid singularities, or keep angles and torques within limits.
	In 1987, Nakamura et al. \cite{Nakamura_IJRR1987} were the first to discuss the concept of
	\emph{task prioritization}: a task was divided into subtasks with different priorities and the
	joint motion was resolved so that lower priority tasks used only redundancy not committed to
	higher priority subtasks.
	They formulated the problem in terms of joint velocity control and joint acceleration control
	\cite{DelPrete_PhDThesis2013}.
	
	The problem of redundancy resolution drew increasingly more attention as research started to
	devote more attention to complex robots, such as humanoids, opening the possibility to perform
	many tasks at the same time \cite{DelPrete_PhDThesis2013}.
	
	During the '80s and '90s new \emph{kinematic control} frameworks, like the ones mentioned by
	Baerlocher and Boulic	\cite{Baerlocher_IROS1998}, generalized the concept of task prioritization
	to the case of an	arbitrary number of tasks.
	However, this kind of approaches were not suited for controlling robots that interact with the
	environment, because they didn't allow for force control \cite{DelPrete_PhDThesis2013}.
	This reason motivated a more recent trend of force control strategies which work on the robot's
	dynamics computing the desired joint torques, by using concepts based on various inverse dynamics
	formulations \cite{Khatib_IJHR2004} \cite{Moro_IJHR2016}.
	
	In 2004, Park et al. \cite{Park_ICRA2004} presented the first implementation of one-link multi-contact
	force control, with demonstration on a 6-DOF PUMA560 manipulator, which was further extended by
	Park and Khatib \cite{Park_ICRA2005} to multiple links in 2005. 
	The robot was able to control three contact forces, distributed on its end-effector and third
	link, while motion was controlled in the remaining three dofs through null space control
	\cite{DelPrete_PhDThesis2013}.
	
	Also in 2004, Khatib et al. \cite{Khatib_IJHR2004} presented the first control framework working on the
	dynamic equations of a humanoid robot, which allowed for \emph{explicit force control} and
	\emph{hybrid control} to be realized, contrary to previous approaches \cite{DelPrete_PhDThesis2013}.
	This one used the \emph{operational space} formalism introduced also by Khatib
	\cite{Khatib_RoboticsAuto1987} which presented a generalized torque/force relationship that
	allowed for a	decomposition of the total torque control signal into two dynamically decoupled
	vectors, corresponding to the task behavior and the posture behavior.
	This formulation made it possible for posture objectives to be controlled without dynamically
	interfering with the operational task, leading to the achievement of whole-body dynamic behavior
	and control	\cite{Khatib_IJHR2004}.
	Also, it allowed to Park and Khatib \cite{Park_ICRA2006} \cite{Park_Robotica2008} to propose a contact
	consistent control framework.
	
	By using the framework of Khatib et al. \cite{Khatib_IJHR2004}, it was also possible to define multiple
	prioritized control objectives and simultaneously control them by providing decoupled closed-loop
	dynamics within their	priority level as explained by Sentis and Khatib \cite{Sentis_IJHR2004}
	\cite{Sentis_ICRA2005} \cite{Sentis_IJHR2005}.
	This prioritized control first allowed for a projection of operational tasks into the constraint
	null-space, and the further establishment of properties among themselves \cite{Sentis_IJHR2005}.
	Then, it was extended by them \cite{Sentis_ICRA2006} to establish a hierarchy between control
	spaces,	assigning top priority to	constraint-handling tasks, while projecting operational tasks
	in the null space of the constraints, and controlling the posture	within the residual redundancy.
	This approach prevents lower priority tasks from interfering with higher priority tasks,
	and provides the means to monitor behavior feasibility at run-time \cite{Sentis_ICRA2006},
	establishing the basis of a methodology for the synthesis of a real-time
	\emph{compliant control of whole-body multi-contact behaviors in humanoid robots}
	\cite{Sentis_PhDThesis2007} \cite{Sentis_MotionPlan2010}.
	
	However, the multi-level task space controller for force-controllable humanoids proposed by
	Sentis and Khatib \cite{Sentis_ICRA2006} didn't mention how to deal with dynamic balancing
	nor walking.
	These topics were addressed (for force-controllable humanoid robots) by the work of Hyon et al.
	\cite{Hyon_IROS2006} \cite{Hyon_ICRA2007} \cite{Hyon_TransRobotics2007}, where a passivity-based
	hierarchical full-body motion controller capable of	rejecting disturbances was proposed,
	and successfully implemented in a full-sized multi-dof humanoid robot.
	This approach, named as \emph{Pasivity-Based Balance Control} (PBBC), was based in the concept of
	gravity compensation, also proposed by them	\cite{Hyon_Humanoids2006}, which makes the closed-loop
	system passive with respect to	additional inputs and external forces when introduced at the lowest
	layer of the controller \cite{Hyon_IROS2006}.
	The method sets a desired applied force from the robot to the environment, then distributes that force
	among predefined contact points, and transforms it to the joint torques directly.
	Furthermore, it does not require contact force measurement or inverse kinematics or dynamics.
	This approach was later extended to consider the integration of a multi-level postural balancing,
	to compensate for yaw perturbations and to provide adaptability to unknown rough terrain,
	as explained by Hyon et al. \cite{Hyon_ICRA2009} \cite{Hyon_TransRobotics2009}
	The controller was also successfully tested on the CB robot \cite{Ott_Humanoids2011}.
	
	In 2010, the \emph{Operational Space Based Control} (OSBC) developed by Sentis et al.
	\cite{Sentis_MotionPlan2010} was further expanded by them \cite{Sentis_TransRobotics2010} to,
	in addition to the force distribution, be able to handle internal forces produced during
	multi-contact interaction tasks	\cite{Ott_Humanoids2011}.
	The approach leverages the \emph{virtual-linkage model}, introduced by Williams and Khatib
	\cite{Williams_ICRA1993}, which provides a physical representation of the internal and CoM
	resultant forces with respect to reaction forces on the supporting surfaces
	\cite{Sentis_TransRobotics2010}.
	This virtual-linkage is a closed chain mechanism that represents the object being manipulated.
	This kinematic structure is composed by actuated prismatic joints connected by passive spherical
	joints, such that each spherical joint be connected to a each grasp point.
	Prismatic joints model internal forces (stresses throughout the object), while the spherical ones
	model internal moments \cite{Williams_ICRA1993}.
	
	\emph{Model-based control methods} can be used to enable fast, dexterous, and compliant motion of
	robots without sacrificing control accuracy \cite{Mistry_ICRA2010}, such that desired joint
	accelerations can be converted into joint torques using inverse dynamics for improved tracking
	performance \cite{Stephens_IROS2010}.
	However, implementing such techniques on floating base robots, e.g., humanoids and legged systems,
	is non-trivial due to comunications delays, under-actuation, dynamically changing constraints from
	the environment, potentially closed loop kinematics, structural compliance, mechanical bias, actuator
	stiction, and a variety of electromechanical phenomena \cite{Mistry_ICRA2010} \cite{Hopkins_ICRA2015}.
	In this context, Mistry et al. \cite{Mistry_ICRA2010} proposed in 2010 a relatively simple technique
	for fully-body model-based control of humanoid robots called \emph{Floating Body Inverse Dynamics}
	(FBID).
	Using an orthogonal decomposition of rigid-body dynamics, they were able to express the complete
	inverse dynamic equations of the robot independently of the contact forces, in order to cope with the
	under-actuation and the dynamically changing contact state inherent in these floating base systems
	\cite{Mistry_ICRA2010}.
	This approach was later extended to address optimal distribution of the contact forces by
	Righetti et al. \cite{Righetti_CLAWAR2010}
	
	Another model-based method known as the \emph{Dynamic Balance Force Control} (DBFC) was also introduced
	in 2010 by Stephens and Atkeson \cite{Stephens_IROS2010}, with the purpose of determining full body
	joint torques based on desired COM motion and contact forces for compliant humanoid robots
	\cite{Stephens_IROS2010}.
	This one uses contact force optimization for balancing the humanoid robot based on the CoM dynamics,
	by combining it with virtual task forces.
	For the mapping of the contact forces to the joint torques, the non-linear full multi-body dynamics are
	considered \cite{Ott_Humanoids2011}.
	However, the input is the desired contact forces.
	Contact forces are computed independent of the full robot model based on a simple COM dynamics model and
	external forces.
	Furthermore, because of the force-based nature of this controller, it can also be modified for the
	compensation of non-contact forces using VMC-like controls \cite{Stephens_IROS2010}.
	
	In 2011, Ott et al. \cite{Ott_Humanoids2011} proposed a balancing controller able to withstand external
	perturbations by distributing the required forces among predefined contact points via a	constrained
	optimization problem, in a similar way as proposed by Hyon et al. \cite{Hyon_TransRobotics2007}
	The approach reused a formulation coming from the field of robot grasping to create a controller that
	keeps both the position and orientation of the robot \cite{Ott_Humanoids2011}.
	This idea was later used by Henze et al. \cite{Henze_IROS2014} which proposed an approach to whole-body
	control for balancing and posture	stabilization of humanoid robots utilizing an optimization of contact
	forces in combination with Model Predictive Control \cite{Henze_IROS2014}.
	
	More recent approaches have explored the idea of simplifying the system dynamic equations by performing
	suitable projections onto the null-space of the contact forces, as proposed by Righetti et al.
	\cite{Righetti_Humanoids2011} based on recent results from analytical dynamics presented by Aghli
	\cite{Aghili_TransRobotics2005}.
	The geometric projector cancels the constraint forces from the system dynamics, removing any need
	of force measurements.
	Also, they are faster to compute than those depending on inertial quantities, as used by Sentis
	\cite{Sentis_PhDThesis2007} and Park \cite{Park_PhDThesis2006}, so resulting in control laws that were
	simpler and computationally more efficient (i.e., a total computation time below 1 ms)
	\cite{DelPrete_IROS2014} \cite{Nori_FrontRobAI2015}.
	Similar methods like Righetti et al. \cite{Righetti_ICRA2011}, and Mistry and Righetti
	\cite{Mistry_RSS2011} presented an in-between approach, extending the	Operational Space formulation
	of Khatib \cite{Khatib_RoboticsAuto1987} to underactuated constrained mechanical systems and proposing
	improved contact models \cite{DelPrete_IROS2014} \cite{Moro_IJHR2016}.
	However, the formulation of Mistry and Righetti	\cite{Mistry_RSS2011} was less efficient than the one
	of Righetti et al. \cite{Righetti_ICRA2011} because it used the inverse of the robot mass matrix
	\cite{DelPrete_IROS2014}.
	This one was later improved by Righetti et al. \cite{Righetti_IJRR2013}
	
	These approaches based on the elimination of the contact forces presented two major drawbacks.
	First, in general they cannot guarantee bounded contact forces.
	Second, every time the robot makes or breaks a contact, the discontinuity in the constraint set
	results in discontinuous control torques.
	These discontinuities may generate jerky movements or, even worse, make the robot slip and fall
	\cite{DelPrete_IROS2014}.
	
	Rather than finding an analytical solution of the control problem, an alternative approach suggested
	by De Lasa et al. \cite{DeLasa_IROS2009} is to use a Quadratic Programming (QP) solver.
	This allows to include inequality constraints into the problem formulation, which can model
	control tasks and physical constraints (e.g. joint limits, friction cones).
	For instance, Saab et al. \cite{Saab_ICRA2011} \cite{Saab_IROS2011} \cite{Saab_TransRobotics2013}
	used inequalities to account for the ZMP conditions on a walking humanoid.
	While this technique is appealing, solving a cascade of QPs with inequality constraints can be critical
	from a computational standpoint.
	Escande et al. \cite{Escande_IJRR2014} reached a computation time of 1 ms on an inverse kinematics
	problem at the price of seldom suboptimal solutions.
	However, they did not consider the inverse-dynamics, which has more than twice the number of variables
	\cite{DelPrete_IROS2014}.
	
	In 2012, Righetti and Schaal \cite{Righetti_Humanoids2012} proposed an inverse dynamics controller for
	a humanoid robot that exploited torque redundancy to minimize any combination of linear and quadratic
	costs in the contact forces and commands.
	Also, Nagasaka et al. \cite{Nagasaka_RobotSymp2012} proposed an elegant formulation of
	\emph{Model Preview Control} (MPC) using the CoM to stabilize a humanoid robot even when it
	operates keeping contact with multiple non-coplanar planes \cite{Audren_IROS2014}
	\cite{Nagasaka_RobotSymp2012}.
	This one was later implemented by Audren et al.	\cite{Audren_IROS2014} in the framework of 
	the multi-contact planner proposed by Escande et al. \cite{Escande_RobAutoSys2013}.
	
	In 2013, Del Prete \cite{DelPrete_PhDThesis2013} introduced a control framework for position and force
	control of constrained floating-base robots denominated as \emph{Task Space Inverse Dynamics}	(TSID)
	\cite{DelPrete_PhDThesis2013} \cite{DelPrete_RobAutoSys2015}.
	Under many aspects, it is similar to other control frameworks, such as the ones proposed by
	De Lasa and Hertzmann \cite{DeLasa_IROS2009}, Mistry and Righetti \cite{Mistry_RSS2011},
	Peters et al. \cite{Peters_AutoRobots2008}, Saab et al. \cite{Saab_ICRA2011}, and Sentis and Khatib
	\cite{Sentis_IJHR2005},	sharing with them most of its limitations \cite{DelPrete_PhDThesis2013}.
	One of the main flaws of the TSID method is that it does not take into account any of the uncertainties
	affecting the humanoid robots: poor torque tracking, sensor noises, delays and model uncertainties.
	As a consequence, the resulting control trajectories may be feasible for the ideal system,
	but not for	the real one.
	In this context, Del Prete and Mansard \cite{DelPrete_RSS2015} proposed to improve the robustness of
	TSID by modeling uncertainties in the joint torques as additive white random noise.
	This resulted in a stochastic optimization problem, in which they could maximize the probability
	to satisfy the inequality constraints (i.e. to be feasible) \cite{DelPrete_RSS2015}.
	
	Also in 2013, Moro \cite{Moro_Humanoids2013} \cite{Moro_IJHR2016} presented an attractor-based
	whole-body motion control	system, developed for	torque-control of floating-base robots performing
	simultaneous tasks.
	The \emph{attractors} are defined as atomic control modules that work in parallel to, and independently
	from the other attractors, generating joint torques that aim to modify the state of the robot so 
	that the error in a target condition is minimized.
	Each task is handled by an attractor, that is associated to a certain physical or derived measure
	(e.g., minimum effort, zero joint momentum, joint limits repeller, end effector position or force).
	One key idea behind of this novel implementation is the control of the effort of the robot without
	directly controlling it.
	The \emph{gravitational stiffness}, a physical quantity that is closely related to the effort,
	is controlled instead, as suggested by him \cite{Moro_Humanoids2015} in a previous work.
	Whenever the gravitational stiffness is maximized, the effort is minimized \cite{Moro_IJHR2016}.
	
	The implementation of hierarchical inverse dynamics controllers based on cascades of quadratic programs
	is not easy as model inaccuracies and real-time computation requirements can be problematic.
	However, in 2014 Herzog et al. \cite{Herzog_IROS2014} were able to evaluate these algorithms experimentally,
	showing that they can be used for feedback control of humanoid robots and that momentum-based
	balance control can be efficiently implemented on a real robot \cite{Herzog_IROS2014}.
	This work (improved later by them \cite{Herzog_Humanoids2015} \cite{Herzog_AutoRobots2016}) also proposed a
	method to simplify the optimization problem by factoring the dynamics equations	of the robot such that it was
	possible to	significantly reduce the computational time and achieve a 1 kHz control loop.
	Another example is the one of Feng et al. \cite{Feng_Humanoids2014} who achieved successful locomotion on
	the Atlas robot using PD feedback in conjunction with torque control.
	That is, these implementations of hierarchical inverse dynamics on purely torque controlled robots have
	required some form of joint-space position and/or velocity feedback to compensate for unmodeled dynamics
	\cite{Hopkins_ICRA2015}.
	
	In 2014, Del Prete et al. \cite{DelPrete_IROS2014} presented a method to control the motion and
	a subset of the contact forces of a floating-base robot, leading to an optimization with reduced
	computational complexity, comparable to inverse-dynamics based approaches \cite{DelPrete_IROS2014}.
	Concurrently, Lack et al. \cite{Lack_ICRA2014} presented a method for achieving planar multi-phase
	multi-contact robot walking using inspired control and optimization \cite{Lack_ICRA2014}.
	
	In 2015 the increasing interest of the research community in multi-contact control frameworks was
	notorious.
	Caron et al. \cite{Caron_RSS2015} proposed the use of a \emph{Gravito-Inertial Wrench Cone} (GIWC)
	as a general multi-contact stability cirterion; that is, a ``ZMP for non-coplanar contacts''.
	Farnioli et al. \cite{Farnioli_ICRA2015} presented a general framework for the quasi-static analysis
	of whole-body loco-manipulation problems.
	This one made use of the \emph{Fundamental Loco-Manipulation Matrix} (FLMM) and its canonical form
	to extract relevant information like the space of the controllable contact forces and the controllable
	displacements of the CoM \cite{Farnioli_ICRA2015}.
	Gams et al. \cite{Gams_Robotica2015} presented a method that allowed real-time motion imitation on
	the COMAN humanoid robot while maintaining stability, based on the prioritized task control proposed
	by Mistry et al. \cite{Mistry_IROS2007}.
	Koenemann et al. \cite{Koenemann_IROS2015} implemented a complete whole-body model-predictive controller
	based on the optimal-control solver MuJoCo developed by Todorov et al. \cite{Todorov_IROS2012}
	This one was implemented in real-time on the physical HRP-2 robot \cite{Koenemann_IROS2015}.
	Kudruss et al. \cite{Kudruss_Humanoids2015} presented a formulation of the reduced multi-contact CoM
	dynamics of a humanoid as an optimal control problem.
	This approach was then used to generate a whole-body motion of climbing stairs with support of a
	handrail, also for the physical HRP-2 robot \cite{Kudruss_Humanoids2015}.
	Liu and Padois \cite{LiuM_IROS2015} proposed a control mechanism to solve whole-body tasks under
	non-rigid contacts which worked online in a reactive way without requiring the knowledge of the
	rigidity of the environment \cite{LiuM_IROS2015}.
	Nori et al. \cite{Nori_FrontRobAI2015} detailed the implementation of the state-of-the-art whole
	body control algorithms on the humanoid robot iCub using a framework based on Del Prete et al.
	\cite{DelPrete_IROS2014}, which allows to control the contact forces, but with a computational
	complexity of the same order of inverse-dynamics-based methods \cite{Nori_FrontRobAI2015}.
	Sherikov et al. \cite{Sherikov_Humanoids2015} introduced a strict prioritization in contact force
	distribution, to reflect situations when an application of certain contact forces should be avoided
	as much as possible, for example, due to a fragility of the support.
	The controller is similar to a previous work also proposed by them \cite{Sherikov_Humanoids2014}
	which had considered a flat ground, improving it to deal with multiple non-coplanar contacts
	\cite{Sherikov_Humanoids2015}.
	
	Also in 2015, and inspired by Koolen et al. \cite{Koolen_Humanoids2013} and Herzog et al.
	\cite{Herzog_IROS2014}, Hopkins et al. \cite{Hopkins_ICRA2015} presented a compliant locomotion
	framework also using model-based whole-body control.
	In order to stabilize the centroidal dynamics during locomotion, they compute the linear momentum
	rate of change objectives using a	novel time-varying controller for the
	\emph{Divergent Component of Motion} (DCM), introduced by	Englsberber et al. \cite{Englsberger_IROS2013}
	and implemented for the first time on hardware on the THOR platform.
	This DCM is a 3D extension of the \emph{Capture Point} (CP) concept, introduced by Pratt et al.
	\cite{Pratt_Humanoids2006}, that simplifies locomotion planning and control on uneven terrain.
	Task-space objectives, including the desired momentum rate of change, were tracked using an efficient
	quadratic program formulation that computed optimal joint torque set-points given frictional contact
	constraints and joint position / torque limits.
	To improve the stability of the proposed inverse dynamics-based approach, joint velocity set-points
	were computed from the optimized joint accelerations and tracked using pre-transmission velocity
	estimates	\cite{Hopkins_ICRA2015}.
	This work was later extended by them \cite{Hopkins_IJHR2016} to include additional material related
	to the low-level control and state estimation implementation for the THOR humanoid as well as
	experimental results and analyssis related to locomotion on grass and gravel.
	
	In 2016, Koolen et al. \cite{Koolen_IJHR2016} presented a momentum-based torque control framework for
	floating-base robots and its application to the Atlas humanoid robot.
	This one was summarily introduced by them \cite{Koolen_Humanoids2013} in a previous work.
	It exploits the fact that the rate of change of whole-body \emph{centroidal momentum}
	is both affine in the joint acceleration vector and linear in the external wrenches applied to the
	robot \cite{Orin_IROS2008}.
	This fact is used to formulate a compact quadratic program, solved at every control time step, which
	reconciles motion tasks in the form of constraints on the joint acceleration vector with the available
	contacts between the robot and its environment in the form of unilateral contacts of force-limited
	grasps.
	Contact forces at unilateral contacts are constrained to remain within conservative polyhedral
	approximations of the expected friction cones \cite{Koolen_IJHR2016}.
	
	\emph{Momentum-Based Control} (MBC) has also gained popularity in recent years.
	Kajita et al. \cite{Kajita_IROS2003} were the first to propose a momentum based control scheme,
	called \emph{Resolve Momentum Control}.
	This scheme used unconstrained least squares to find joint velocity references to be tracked by
	a low level controller.
	The relationship between momentum and joint velocities was later studied in more detail by
	Orin and Goswami \cite{Orin_IROS2008}.
	Based on this work, they presented a momentum-based control framework for torque-controlled robots
	that has been used by Lee and Goswami	\cite{LeeS_IROS2010} \cite{LeeS_AutoRobots2012} and
	Orin et al. \cite{Orin_AutoRobots2013}
	These works then inspired the framework proposed by Hopkins et al. \cite{Hopkins_IJHR2016} and
	Koolen et al. \cite{Koolen_IJHR2016}.
	
	These frameworks have required specialized computational algorithms to compute the
	\emph{Centroidal Momentum Matrix} (CMM) and its derivative, which relate rates of change in
	centroidal momentum to joint rates and accelerations of the humanoid.
	However, these specialized algorithms are not always required.
	Wensing and Orin \cite{Wensing_IJHR2016} showed that since the dynamics of the centroidal momentum
	are embedded in the joint-space dynamics equation of motion, the CMM and terms involving its
	derivative can be computed from the joint-space mass matrix and Coriolis terms.
	This approach presented improvements in terms of its generality, compactness, and efficiency in
	comparison to previous specialized algorithms~\cite{Wensing_IJHR2016} and expanded previous
	works developed by them \cite{Wensing_ICRA2013} \cite{Wensing_IJHR2013}.
	
	A timeline of the works presented here is shown in Table~\ref{tab:Timeline}.
	
	\begin{table*}[t]
	%
	\caption{Timeline for works in multi-contact control.}
	\label{tab:Timeline}
	\scriptsize
	\centering
	%
	\begin{tabular}{|l|c|c|c|c|c|c|c|c|c|c|c|c|c|}
		\hline
		First Author	&
		2004	& 2005	& 2006	& 2007	& 2008	& 2009	& 2010	& 2011	& 2012	& 2013	& 2014	& 2015	& 2016	\\
		\hline
		Khatib O.			&
		\cite{Khatib_IJHR2004}	& & & & & & & & & & & & \\
		Park J.				&
		\cite{Park_ICRA2004}		& \cite{Park_ICRA2005}	& \cite{Park_ICRA2006} \cite{Park_PhDThesis2006}
		&	&	\cite{Park_Robotica2008}	& & & & & & & & \\
		Sentis L.			&
		\cite{Sentis_IJHR2004} & \cite{Sentis_ICRA2005} \cite{Sentis_IJHR2005}	& \cite{Sentis_ICRA2006}
		& \cite{Sentis_PhDThesis2007}	& & \cite{Sentis_IROS2009}
		& \cite{Sentis_MotionPlan2010} \cite{Sentis_TransRobotics2010}	&	& & & & & \\
		Hyon S.				&
		& & \cite{Hyon_Humanoids2006} \cite{Hyon_IROS2006}	& \cite{Hyon_ICRA2007} \cite{Hyon_TransRobotics2007}
		& & \cite{Hyon_ICRA2009} \cite{Hyon_TransRobotics2009}	& & & & & & & \\
		Mistry M.			&
		& & & \cite{Mistry_IROS2007} & & & \cite{Mistry_ICRA2010}	& \cite{Mistry_RSS2011}	& & & & & \\
		Orin D.				&
		& & & & \cite{Orin_IROS2008} & & & & & \cite{Orin_AutoRobots2013} & & & \\
		Peters J.			&
		& & & & \cite{Peters_AutoRobots2008} & & & & & & & & \\
		De Lasa M.		&
		& & & & & \cite{DeLasa_IROS2009}	& & & & & & & \\
		Stephens B.		&
		& & & & & & \cite{Stephens_IROS2010}	& & & & & & \\
		Lee S.				&
		& & & & & & \cite{LeeS_IROS2010}	& & \cite{LeeS_AutoRobots2012}	& & & & \\
		Righetti L.		&
		& & & & & & \cite{Righetti_CLAWAR2010} & \cite{Righetti_ICRA2011} \cite{Righetti_Humanoids2011}
		& \cite{Righetti_Humanoids2012}	& \cite{Righetti_IJRR2013}	& & & \\
		Ott C.				&
		& & & & & & & \cite{Ott_Humanoids2011}	& & & & & \\
		Saab L.				&
		& & & & & & & \cite{Saab_ICRA2011} \cite{Saab_IROS2011}	& & \cite{Saab_TransRobotics2013}	& & & \\
		Nagasaka K.		&
		& & & & & & & & \cite{Nagasaka_RobotSymp2012}	& & & & \\
		Koolen T.			&
		& & & & & & & & & \cite{Koolen_Humanoids2013} & & & \cite{Koolen_IJHR2016}	\\
		Escande A.		&
		& & & & & & & & & \cite{Escande_RobAutoSys2013} & & & \\
		Del Prete A.	&
		& & & & & & & & & \cite{DelPrete_PhDThesis2013}	& \cite{DelPrete_IROS2014}
		& \cite{DelPrete_RSS2015} \cite{DelPrete_RobAutoSys2015}	& \\
		Moro F.					&
		& & & & & & & & & \cite{Moro_Humanoids2013} & & \cite{Moro_Humanoids2015} & \cite{Moro_IJHR2016}	\\
		Wensing P.			&
		& & & & & & & & & \cite{Wensing_ICRA2013} \cite{Wensing_IJHR2013}	& & & \cite{Wensing_IJHR2016}	\\
		Audren H.			&
		& & & & & & & & & & \cite{Audren_IROS2014}	& & \\
		Herzog A.			&
		& & & & & & & & & & \cite{Herzog_IROS2014} & \cite{Herzog_Humanoids2015}
		& \cite{Herzog_AutoRobots2016}	\\
		Feng S.				&
		& & & & & & & & & & \cite{Feng_Humanoids2014}	& & \\
		Lack J.				&
		& & & & & & & & & & \cite{Lack_ICRA2014}	& & \\
		Caron S.			&
		& & & & & & & & & & & \cite{Caron_RSS2015}	& \\
		Farnioli E.		&
		& & & & & & & & & & & \cite{Farnioli_ICRA2015}	& \\
		Gams A.				&
		& & & & & & & & & & & \cite{Gams_Robotica2015}	& \\
		Koenemann J.	&
		& & & & & & & & & & & \cite{Koenemann_IROS2015}	& \\
		Kudruss M.		&
		& & & & & & & & & & & \cite{Kudruss_Humanoids2015}	& \\
		Liu M.				&
		& & & & & & & & & & & \cite{LiuM_IROS2015}	& \\
		Nori F.				&
		& & & & & & & & & & & \cite{Nori_FrontRobAI2015}	& \\
		Sherikov A.		&
		& & & & & & & & & & & \cite{Sherikov_Humanoids2015}	& \\
		Hopkins M.			&
		& & & & & & & & & & & \cite{Hopkins_ICRA2015}	& \cite{Hopkins_IJHR2016}	\\
		\hline
	\end{tabular}
	%
\end{table*}
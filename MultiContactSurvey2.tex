\documentclass[conference]{IEEEtran}

\usepackage{graphicx}

\usepackage[tbtags]{amsmath}
\usepackage{amsfonts}

\usepackage{paralist}

\usepackage{url}

\DeclareGraphicsExtensions{.pdf,.png,.jpg}

\renewcommand{\vec}[1]{\boldsymbol{#1}}
\newcommand*{\R}[1]{\mathbb{R}^{#1}}

\def\figurename{Fig.}

\begin{document}

	\title{\LARGE \bf Research Survey on Multi-Contact Force Control}

	\author{
		
		\IEEEauthorblockN{
			Rafael CISNEROS,
			Mitsuharu MORISAWA,
			Fumio KANEHIRO} \\
			
		\IEEEauthorblockA{
			Humanoid Research Group, Intelligent Systems Institute, \\
			National Institute of Advanced Industrial Science and Technology (AIST), Japan}}
  
	\maketitle

	\thispagestyle{empty}
	\pagestyle{empty}
	
	\begin{abstract}
		Recently, there is a growing interest with respect to the use of humanoid robots in activities
		involving the interaction with unstructured environments and their ability to perform more
		complex tasks.
		In order to achieve that, it is required to consider that the humanoid robot will make contact
		with this environment not only by using its feet, but by using its hands or any other point on
		its body, and to do it compliantly for its own safety and the safety of the people that it is
		expected to interact with; that is, to exhibit a multi-contact compliant behavior.
		This paper surveys the state of the art in whole-body multi-contact force control, by means
		of comparing the different technologies proposed by the robotics research	community.
	\end{abstract}
	
	\section{Introduction}
	\label{sec:introduction}

	\subsection{Motivation: position control vs. force control}
		\label{sub:motivation}
		
		Robotic control is most frequently accomplished with a \emph{stiff position control} system
		\cite{Khatib_ICRA2008}, which tries to follow a desired position trajectory considering external
		forces as disturbances \cite{DelPrete_PhDThesis2013}.
		This typical controller is realized at the joint level using PD control with high gains.
		Thus, each joint is treated	independently	and, since the controller cannot account for the dynamics
		of the system, coupled dynamic effects between joints are ignored and treated as a disturbances.
		This approach has been well suited in factory automation, where joint control is a dominant
		paradigm in \emph{industrial robots} \cite{Khatib_ICRA2008}.
		
		Similar to industrial robots, \emph{humanoid robots} have been traditionally based on joint
		position control, which in combination with a stiff and precisely manufactured drive train
		leads to a high achievable precision \cite{Englsberger_Humanoids2015}.
		However, this type of control generally exhibits \emph{high impedance}, and the speed at which
		it will comply to an unknown force is restricted \cite{Stephens_IROS2010}.
		Furthermore, it is also severely limited in performing advanced robotic tasks, especially in
		unstructured dynamic complex environments \cite{Khatib_ICRA2008}, where it is important for the
		robot not only to adapt	quickly to sudden environmental changes or disturbances, especially to
		unknown external forces	\cite{Hyon_TransRobotics2007}, but also to interact closely with people.
		On the other hand, robots with \emph{low impedance} joints can comply faster.
		This is useful, but also makes balance control more important and more difficult \cite{Stephens_IROS2010}.
		That insufficiency is an important disadvantage for position-based controllers
		in terms of force	interaction \cite{Hyon_TransRobotics2007}.
		
		In a \emph{position-based compliance control} the robot adapts compliantly to external forces by using a
		force sensor.
		Such methods however, can deal only with forces via \emph{virtual compliance}; for that reason,
		time delays are, in principle, unavoidable.
		Moreover, the controller requires force sensors at every expected contact point.
		Without them, the robot cannot adapt to unknown external forces.
		Although it is possible to install force sensors over the whole body,
		this imparts a heavy computational burden because of the number of contact Jacobians.
		Also, it is difficult to assign appropriate weights.
		The best way to treat unknown disturbances and external forces is to transform desired contact
		(interaction) forces into joint torques directly, even though controlling these joint torques
		precisely is technically challenging \cite{Hyon_TransRobotics2007}.
		
		One alternative approach for addressing this problem(s)	is to use
		\emph{joint torque sensing and control}	\cite{Khatib_ICRA2008} \cite{Ott_Humanoids2011}.
		This kind of control has recently attracted increased attention, mainly due	to the fact that
		it is possible to interact robustly with the environment, as well as exhibit a safe
		and compliant behavior while interacting with humans (probably the most crucial issue) and
		in case of self-collisions \cite{Englsberger_Humanoids2015}.
		
	\subsection{Humanoids: from balance to multi-contact force control}
		\label{sub:humanoid_robots}
		
		Research and development of humanoid robots started in 1973 with the development of WABOT-1
		in Waseda	University, and the first successful walking humanoid system presented by Honda in
		1995 \cite{Park_PhDThesis2006}.
		Since then, the research had mainly focused on the hardware design and the realization of some
		basic	motions such as walking and balancing \cite{Harada_ICRA2003}.	
		
		Humanoid balance depends critically on controlling the linear and angular momentum of the	system,
		quantities that can be directly controlled by contact forces \cite{Stephens_IROS2010}.
		In general, a \emph{contact} can be seen as a continuum of infinitesimal forces acting on
		the surface of a rigid body, being the effect of contact forces represented with an
		equivalent wrench, composed by a three-dimensional force and a three-dimensional torque.
		\cite{Nori_FrontRobAI2015}.
		
		Given a robot with stiff joint position control and a known environment, the most common approach
		to balance is to generate a stable trajectory of the COM and then track it using inverse kinematics
		(IK) \cite{Stephens_IROS2010}.
		These trajectories can be modified online to produce whole body balance in the presence of small
		disturbances \cite{Sugihara_IROS2002}.
		For environments with small uncertainty or small disturbances, the inverse kinematics can be modified
		to directly control the contact forces using force feedback, a method known as Inverse Kinematics
		Force Control (IKFC) \cite{Fujimoto_ICRA1996}.
		
		Since the kinematical structure of a humanoid robot is similar to that of a human, a humanoid
		robot	is expected to work in the same environment as the human.
		Then, to accomplish the required tasks under such an environment, it should be considered that
		a	humanoid robot manipulates an object by means of a cooperation between arms and legs
		\cite{Harada_ICRA2003}; that is, due to multi-contact interactions.
		
		One of the first works considering the arm / leg coordination of a humanoid robot is the one of
		Inoue et al. \cite{Inoue_ICRA2000}, which determined the posture of a humanoid robot taking the
		manipulability of the arms into consideration \cite{Harada_IROS2003}.
		In this work, an impedance control was implemented to make the arm tips always follow their desired
		position, while considering	an external force	being applied \cite{Inoue_ICRA2000}.
		
		Forces were also considered within a force-balanced biped robot (not for multi-contact control)
		by Pratt et al. \cite{Pratt_IJRR2001}
		This work introduces the Virtual Model Control (VMC), a motion control framework that uses
		simulations of virtual components to create virtual forces generated when the virtual components
		interact with a robotic system.
		This is the simplest method that only uses a kinematic model \cite{Stephens_IROS2010}.
		
		The first research considering a pushing manipulation by a humanoid robot
		(and the influence of	the corresponding external forces in a multi-contact scenario)
		was reported by Harada et al. \cite{Harada_ICRA2003}
		These authors considered an extension to the Zero Moment Point (ZMP) dynamic-evaluation
		criterion, defined as the Generalized Zero Moment Point (GZMP), in order to enable generalized
		multicontact-locomotion behaviors.
		This extension was later used by them \cite{Harada_IROS2003} for arm / leg coordination.
		
		A formulation of multi-contact compliant motion control was, in our knowledge, first addressed
		by Park et al. \cite{Park_ICRA2004}
		This one was based on the contact model described by Featherstone et al. \cite{Featherstone_ICRA1999}
		together with the operational space formalism introduced by Khatib \cite{Khatib_RoboticsAuto1987},
		which made it possible to use force control to attain compliant contacts.
		Although the target of this work was not a humanoid robot, it addressed the case in which one
		link of a manipulator	was making contact with the environment in more than one point, providing
		the basics for future humanoid-related research in multi-contact compliant motion control.
		
	\subsection{Challenges in real-time force control of humanoid robots}
		\label{sub:challenges}
		
		The real-time multi-contact force control of a humanoid robot involves the optimal distribution of an
		undetermined force system subjected to physical constraints \cite{Chen_MIRC1999}
		\cite{Nahon_TransRoboticsAuto1992}.
		\emph{Force distribution} is the inverse dynamics problem for multiple-chain systems of links (arms,
		legs, or fingers) supporting a load, body or object, in which the motion is completely specified and
		the internal forces / torques to effect	this motion are to be determined \cite{Orin_AdvRobotics1989}.
		However, the solution to these \emph{inverse dynamics} equations of a multi-legged system is not unique
		(it is an ill-posedness problem),	but it can be chosen in an optimal manner by introducing an
		objective function \cite{Hyon_TransRobotics2007} \cite{Chen_MIRC1999}.
		
		Physical interaction with the environment definitely represents a crucial issue:
		while we can estimate a dynamical model of the robot, it is much harder to retrieve a model of the
		\emph{physical constraints} (the \emph{contact dynamics}), for which simplified models are usually
		used \cite{DelPrete_PhDThesis2013}.
		However, even though some of these ones can be simplified and represented by equality constraints,
		many more can only be represented mathematically as inequalities due to the nature of the
		contacts involved, i.e., normal contact forces between the feet and the ground cannot be negative,
		and, as supporting feet should not slip, the magnitude of the tangential force at each foot cannot
		exceed the maximum force of	static friction.
		In addition, the torque of each joint must lie within the allowed range.
		Therefore, force distribution requires to find a solution for a nonlinear programming problem
		with inequality constraints, a problem that should be solved in real-time, although it is difficult
		to do so due to the computational cost \cite{Chen_MIRC1999}.
		This is because the results of this force distribution are used, after an appropriate force-to-torque
		transformation, to provide torque set points for the actuators \cite{Nahon_TransRoboticsAuto1992}.

	\section{Areas of inspiration}
	\label{sec:inspiration}
	
	Force distribution due to multi-contact interactions in \emph{multi-legged walking robots} has been
	addressed since the early 80's.
	Then, research began to focus on modeling multi-grasp behaviors in \emph{dexterous mechanical hands}
	and the associated internal forces acting between \emph{multiple cooperating manipulators};
	that is, to a number of simple closed	multiple-chain robotic systems \cite{Orin_AdvRobotics1989}
	\cite{Nahon_TransRoboticsAuto1992} \cite{Chen_MIRC1999} \cite{Sentis_TransRobotics2010}.
	These fields provided the inspiration and tools that were later used to implement	multi-contact force
	control in humanoid robots.
	
	\subsection{Multi-legged walking robots}
		\label{sub:walking_robots}
		
		Some of the earliest works in the area of multilegged walking robots was the one of Orin and Oh
		\cite{Orin_DSMC1981} in 1981, where a solution that	optimized a weighted combination of energy
		consumption and load balance was used on simple closed-chain mechanisms in which a single member,
		called the reference member, is supported by several chains.
		While the model included all three types of robotics systems of interest, it was applied to	multilegged
		vehicles only, specifically to drive a hexapod locomotion vehicle.
		Hard point contact with friction was used to model the foot / support-surface interaction.
		Joint actuator limits and leg dynamic effects were also included in the formulation.
		However, the method used to include these was computationally inefficient.
		The above statement was done by Orin and Chen \cite{Orin_AdvRobotics1989}, which in turn presented a
		computationally efficient formulation of the force distribution problem which included the dynamic
		effects of the chains and physical limits on their actuators, as well as used a contact
		modeling relatively general and capable of handling hard point contact, soft finger contact,
		or rigid contact with an irregular-shaped	object or with uneven terrain.
		
		Later, at the beginning of the 90's, Klein and Kittivatcharapong \cite{Klein_RoboticsAuto1990}
		proposed to solve the force distribution problem for the limbs of a legged vehicle with friction
		cone constraints, while Kumar and Waldron \cite{Kumar_MechDesign1990} addressed the problem of the
		appropriate distribution of forces between the legs of a legged locomotion system for walking
		on uneven terrain.
		Also, Gardner and Srinivasan \cite{Gardner_DSMC1990} solved the force distribution
		problem in a closed form for a walking machine such that it would be computationally simpler,
		and after Gardner \cite{Gardner_DSMC1991} extended the previous work to allow for
		arbitrarily oriented surface normals at the point of contact between the feet and the ground.
		
		At the end of the 90's Liu and Wen \cite{LiuH_RoboticSystems1997} focused on preventing leg
		slippage by means of an efficient approach to optimize the foot force distribution on a quadruped
		walking vehicle.
		Later, Chen et al. \cite{Chen_MIRC1999} developed a real-time force control for a
		quadruped robot by transforming the friction constraints from nonlinear inequalities into a
		combination of linear	equalities and linear inequalities reducing the problem size, and
		Hung et al. \cite{Hung_SysManCyb2000} presented a systematic formulation of the force
		distribution equations for a general tree-structured robot mechanism.
		
	\subsection{Multiple cooperating manipulators}
		\label{sub:cooperating_manipulators}
	
		In the field of cooperation (or coordination) of multiple manipulators some representative works
		that date from the 90's also made use of optimal force distribution.
		For example, Shin and Chung \cite{Shin_IROS1991} proposed a method called weak point minimization
		applicable to weakly connected assembly parts and weak joints in cooperating multiple robots.
		Choi et al. \cite{Choi_ICRA1992} found an optimal load distribution for two cooperating robots by
		utilizing a force ellipsoid, a concept that was also used together with the manipulability ellipsoid
		by them later \cite{Choi_Robotica1993}.
		Also, Kown and Lee \cite{Kwon_SICE1996} proposed a compact dual method for multiple cooperating
		robots, and after they \cite{Kwon_IntellRobotSys1998} used quadratic constraints to reduce their
		number and improve efficiency.
		Finally, Featherstone et al. \cite{Featherstone_ICRA1999} presented a general first-order kinematic
		model of	frictionless rigid-body contact for use in hybrid force / motion control that made it
		possible to include multiple points of contact.
		
	\subsection{Dexterous mechanical hands}
		\label{sub:mechanical_hands}
		
		As for grasping with dexterous mechanical hands, there are also some representative works published
		also around the 90's.
		Cheng and Orin \cite{Cheng_ICRA1989} \cite{Cheng_TransRoboticsAuto1990} used the duality theory of
		linear programming to obtain the general solution of linear equality constraints and applied it to
		a grasping system, and later they \cite{Cheng_TransSysManCyb1991a} \cite{Cheng_TransSysManCyb1991b}
		proposed a computationally efficient formulation applicable to multiple-chain robotic systems.
		Mukherjee and Waldron \cite{Mukherjee_MechDesign1992} proposed to minimize the maximum value of
		the friction angle at the points of contact of a three-fingered hand.
		Later, Chen \cite{Chen_ICRA1995} analyzed the mechanics of grasping general solid objects under a
		frictional point contact model and proposed to balance disturbances by means of a particular set
		of internal forces, while Cheng \cite{Cheng_SysManCyb1997} proposed an efficient method for obtaining
		the general solution for the force balance equations with hard point contacts.
		Also, Liu \cite{LiuY_TransRoboticsAuto1999} formalized a qualitative test of 3D frictional
		form-closure grasps, and Al-Gallaf \cite{AlGallaf_RobAutoSys2006} presented a novel neural
		network for dexterous hand-grasping inverse kinematics mapping used in force optimization.
		
	\subsection{Non-specialized field of application}
		\label{sub:non_specialized}
		
		Some works were not mainly focused on one applicable field of the optimal force distribution,
		but on its application to closed kinematic chains, comprising the above stated fields.
		For example, Nahon and Angeles \cite{Nahon_TransRoboticsAuto1992} focused on solving the
		optimization problem with equality and inequality constraints in order to achieve real-time
		control of cooperating manipulators, mechanical hands and walking machines.
		In contrast, Sreenivasan et al. \cite{Sreenivasan_MechDesign1996} used the redundancy in actuation
		of multifingered hands and walking vehicles to optimize the force distribution, but the developed
		algorithm was not suited for real-time implementation, just for offline planning.
	
	\section{Contact Force Estimation}
	\label{sec:contact_estimation}
	
	In most robotics applications, robots come into contact with the environment just with
	their end-effectors.
	Nevertheless, robots could benefit from making contacts with other parts of their bodies,
	in the same way humans do.
	This lack is mainly due to the fact that nowadays most robots do not have an artificial skin
	that allows them to detect and localize contacts.
	Either joint torque sensors or 6 axis Force/Torque (F/T) sensors are usually used to provide
	contact	feed-back to robots.
	However, with these sensors it is not possible to retrieve the exact contact location,
	unless we make strong assumptions about the contact
	(e.g. zero moment applied, known force direction).
	Moreover, the wrench (i.e. force and moment) that is applied at the contact point cannot be
	measured, unless the contact location is somewhat known \cite{DelPrete_PhDThesis2013}.
	
	Within this context, Sentis et al. \cite{Sentis_IROS2009} presented a theoretical framework
	to model and control robots that are subject to multiple contacts, but the authors didn't
	discuss how to localize the contacts and estimate the	contact forces / pressures.
	On the other hand, Park and Khatib \cite{Park_ICRA2005} presented a compliant motion control
	framework for multiple contacts and they tested it with a PUMA560 manipulator.
	In the experiments the geometry and the stiffness of the environment were assumed to be known
	a priory in order to compute the contact points and the contact forces.
	
	A probabilistic approach was later proposed by Petrovskaya et al. \cite{Petrovskaya_ICRA2007},
	in which the authors used	an active sensing strategy to estimate at the same time the shape of
	the robot and the contact point.
	Nevertheless, they tested the method only with extremely simple geometries of robot and
	environment (the environment was a point and the robot link was a line) and the authors said
	that,	for more complex geometries, more sophisticated active exploration strategies would likely
	be needed \cite{DelPrete_PhDThesis2013}.
	
	When both tactile sensors and F/T sensors are available, it is possible to estimate contact
	locations and	contact wrenches, and so implement reliable contact force control.
	This was the approach taken by Del Prete \cite{DelPrete_PhDThesis2013} who implemented the
	corresponding method on the iCub humanoid	robot \cite{Metta_PerMIS2008}, an open robotic
	platform designed for studying embodied cognition featuring 6-axis F/T sensors and a distributed
	sensorized skin.

	\section{Force Control}
	\label{sec:force_control}
	
	\subsection{Paradigms}
		\label{sub:paradigms}
		
		There are different approaches (or paradigms) to force control that we can find in the
		literature \cite{DelPrete_PhDThesis2013}:
		%
		\begin{itemize}
			
			\item \emph{Explicit force control}.
						This one exploits force feedback to adjust the
						contact force to a desired value \cite{Volpe_TransAutoControl1993}.
			
			\item \emph{Implicit force control}.
						This one, which without any force feedback, regulates the position of the
						contact point, while tuning the joint servo gains so as	to give a particular
						stiffness / admittance to the contact point	\cite{Rocco_Automatica1997}.
			
			\item \emph{Impedance control}.
						Proposed originally by Hogan \cite{Hogan_AmericanCtrlConf1984}.
						This one consists in the control of the motion of the	manipulator and,
						in addition, its dynamic behavior in response to external	forces.
						Based on the observation that all environments can be modeled as
						\emph{admittances} (i.e. physical systems that accept force inputs
						and yield motion outputs), Hogan suggested to control the manipulator as
						an \emph{impedance} (i.e. physical system that accepts motion inputs
						and yield force outputs).
						This guarantees that thew two dynamically interacting systems physically
						complement each other.
						Different from previous approaches, impedance control does not regulate
						motion or force directly, but instead it regulates the ratio of force to
						motion, that is the mechanical impedance \cite{DeShutter_ControlProblemsRobAuto1998}.
			
			\item \emph{Hybrid control}.
						Introduced by Craig and Raibert \cite{Craig_SoftwareApp1979}.
						This one regulates force and position independently, along orthogonal directions.
						Any assembly task involving rigid frictionless contacts defines two sets of
						\emph{natural constraints}: position constraints prevent the robot from moving
						through the environment, whereas force constraints prevent the application of
						forces along the tangent directions.
						These two sets of constraints partition the space into two orthogonal subspaces,
						which are then controlled according to different criteria
						\cite{Khatib_RoboticsAuto1987}.
						
			\item \emph{Hybrid impedance control}.
						Introduced by Anderson and Spong \cite{Anderson_JRobticsAuto1988}.
						This one combines impedance control and hybrid position / force control into one
						strategy, while allowing for more sophisticated impedances.
						
			\item \emph{Parallel control}.
						Proposed by Chiaverini and Sciavicco \cite{Chiaverini_RoboticsAuto1993}.
						It consists on superimposing a proportional-derivative position control on a
						proportional-integral force control.
						The integral action in the force regulator ensures dominance of the force control
						over the position control, which is typically desired for safety reasons.
						
		\end{itemize}
		
	\subsection{Force distribution}
		\label{sub:distribution}
		
		Force control approaches for multiple-chain robotic systems have solved the	force distribution
		problem by means of a variety of methods, once the equations had been fully	derived
		\cite{Orin_AdvRobotics1989}.
		Some of the representative ones, as identified by Orin and Cheng \cite{Orin_AdvRobotics1989} and
		Chen et al. \cite{Chen_MIRC1999},	are:
		
		\begin{enumerate}
		
			\item \emph{Linear-Programming (LP) Method}.
						The problem is formulated as a linear-programming problem, by replacing the
						friction cone with a piecewise linear pyramid so as to express friction constraints by
						linear inequalities.
						However, this method is difficult to be applied to complex systems in real-time, as it leads
						easily to discontinuous solutions even under smooth changes in constraints
						\cite{Nahon_TransRoboticsAuto1992}.
						
			\item \emph{Gradient Projection Method}.
						This one is useful for considering unilateral constraints in a cost function whose
						gradient is projected in the space of free motion as a lowest-priority task
						\cite{Mansard_TransRobotics2009}.
						
			\item \emph{Compact-Dual LP (CDLP) Method}.
						This one results in a smaller problem size by using the compact-dual linear programming,
						but it cannot overcome the discontinuous problem \cite{Nahon_TransRoboticsAuto1992}.
						
			\item \emph{Quadratic Programming (QP) Method}.
						It is superior to LP and CDLP in terms
						of the quality of the solution and also can be implemented in real-time because the time for
						obtaining a solution does not depend on an initial guess~\cite{Chen_MIRC1999}.
						
			\item \emph{Non-Linear Programming (NLP) Method}.
						NLP can be used when the program can be difficult to solve if the local QP is	a poor estimate
						of the true nonlinear program \cite{Posa_WAFR2012}.
						However, it is time consuming and there is no mention about real-time
						implementation~\cite{Sharma_NUiCONE2011}.
						
			\item \emph{Analytical Method}.
						This one is mainly applied to walking robots.
						It is characterized by finding a relation among feet forces in order to prevent legs from
						slipping, and consequently adding equality constraints so that the undetermined force system
						may be transformed into a determined system after combining inverse dynamic equations with the
						equality constraints.
						However, it attempts only to prevent leg slippage, neglecting the other inequality constraints
						\cite{Chen_MIRC1999}.
		
		\end{enumerate}
	
	\section{Multi-task Whole-body Control Frameworks}
	\label{sec:whole_body}
	
	In order to meet the increasing demand of flexibility and versatility, robot manipulators
	with more than six degrees-of-freedom (dof) appeared.
	These manipulators introduced the problem of \emph{redundancy resolution}, namely to select
	a unique solution among the infinite control actions that generate the desired end-effector
	behavior \cite{DelPrete_PhDThesis2013}.
	
	At first, redundancy was exploited to minimize the kinetic/potential energy of the manipulator,
	avoid obstacles in the workspace, avoid singularities, or keep angles and torques within limits.
	In 1987, Nakamura et al. \cite{Nakamura_IJRR1987} were the first to discuss the concept of
	\emph{task prioritization}: a task was divided into subtasks with different priorities and the
	joint motion was resolved so that lower priority tasks used only redundancy not committed to
	higher priority subtasks.
	They formulated the problem in terms of joint velocity control and joint acceleration control
	\cite{DelPrete_PhDThesis2013}.
	
	The problem of redundancy resolution drew increasingly more attention as research started to
	devote more attention to complex robots, such as humanoids, opening the possibility to perform
	many tasks at the same time \cite{DelPrete_PhDThesis2013}.
	
	During the '80s and '90s new \emph{kinematic control} frameworks, like the ones mentioned by
	Baerlocher and Boulic	\cite{Baerlocher_IROS1998}, generalized the concept of task prioritization
	to the case of an	arbitrary number of tasks.
	However, this kind of approaches were not suited for controlling robots that interact with the
	environment, because they didn't allow for force control \cite{DelPrete_PhDThesis2013}.
	This reason motivated a more recent trend of force control strategies which work on the robot's
	dynamics computing the desired joint torques, by using concepts based on various inverse dynamics
	formulations \cite{Khatib_IJHR2004} \cite{Moro_IJHR2016}.
	
	In 2004, Park et al. \cite{Park_ICRA2004} presented the first implementation of one-link multi-contact
	force control, with demonstration on a 6-DOF PUMA560 manipulator, which was further extended by
	Park and Khatib \cite{Park_ICRA2005} to multiple links in 2005. 
	The robot was able to control three contact forces, distributed on its end-effector and third
	link, while motion was controlled in the remaining three dofs through null space control
	\cite{DelPrete_PhDThesis2013}.
	
	Also in 2004, Khatib et al. \cite{Khatib_IJHR2004} presented the first control framework working on the
	dynamic equations of a humanoid robot, which allowed for \emph{explicit force control} and
	\emph{hybrid control} to be realized, contrary to previous approaches \cite{DelPrete_PhDThesis2013}.
	This one used the \emph{operational space} formalism introduced also by Khatib
	\cite{Khatib_RoboticsAuto1987} which presented a generalized torque/force relationship that
	allowed for a	decomposition of the total torque control signal into two dynamically decoupled
	vectors, corresponding to the task behavior and the posture behavior.
	This formulation made it possible for posture objectives to be controlled without dynamically
	interfering with the operational task, leading to the achievement of whole-body dynamic behavior
	and control	\cite{Khatib_IJHR2004}.
	Also, it allowed to Park and Khatib \cite{Park_ICRA2006} \cite{Park_Robotica2008} to propose a contact
	consistent control framework.
	
	By using the framework of Khatib et al. \cite{Khatib_IJHR2004}, it was also possible to define multiple
	prioritized control objectives and simultaneously control them by providing decoupled closed-loop
	dynamics within their	priority level as explained by Sentis and Khatib \cite{Sentis_IJHR2004}
	\cite{Sentis_ICRA2005} \cite{Sentis_IJHR2005}.
	This prioritized control first allowed for a projection of operational tasks into the constraint
	null-space, and the further establishment of properties among themselves \cite{Sentis_IJHR2005}.
	Then, it was extended by them \cite{Sentis_ICRA2006} to establish a hierarchy between control
	spaces,	assigning top priority to	constraint-handling tasks, while projecting operational tasks
	in the null space of the constraints, and controlling the posture	within the residual redundancy.
	This approach prevents lower priority tasks from interfering with higher priority tasks,
	and provides the means to monitor behavior feasibility at run-time \cite{Sentis_ICRA2006},
	establishing the basis of a methodology for the synthesis of a real-time
	\emph{compliant control of whole-body multi-contact behaviors in humanoid robots}
	\cite{Sentis_PhDThesis2007} \cite{Sentis_MotionPlan2010}.
	
	However, the multi-level task space controller for force-controllable humanoids proposed by
	Sentis and Khatib \cite{Sentis_ICRA2006} didn't mention how to deal with dynamic balancing
	nor walking.
	These topics were addressed (for force-controllable humanoid robots) by the work of Hyon et al.
	\cite{Hyon_IROS2006} \cite{Hyon_ICRA2007} \cite{Hyon_TransRobotics2007}, where a passivity-based
	hierarchical full-body motion controller capable of	rejecting disturbances was proposed,
	and successfully implemented in a full-sized multi-dof humanoid robot.
	This approach, named as \emph{Pasivity-Based Balance Control} (PBBC), was based in the concept of
	gravity compensation, also proposed by them	\cite{Hyon_Humanoids2006}, which makes the closed-loop
	system passive with respect to	additional inputs and external forces when introduced at the lowest
	layer of the controller \cite{Hyon_IROS2006}.
	The method sets a desired applied force from the robot to the environment, then distributes that force
	among predefined contact points, and transforms it to the joint torques directly.
	Furthermore, it does not require contact force measurement or inverse kinematics or dynamics.
	This approach was later extended to consider the integration of a multi-level postural balancing,
	to compensate for yaw perturbations and to provide adaptability to unknown rough terrain,
	as explained by Hyon et al. \cite{Hyon_ICRA2009} \cite{Hyon_TransRobotics2009}
	The controller was also successfully tested on the CB robot \cite{Ott_Humanoids2011}.
	
	In 2010, the \emph{Operational Space Based Control} (OSBC) developed by Sentis et al.
	\cite{Sentis_MotionPlan2010} was further expanded by them \cite{Sentis_TransRobotics2010} to,
	in addition to the force distribution, be able to handle internal forces produced during
	multi-contact interaction tasks	\cite{Ott_Humanoids2011}.
	The approach leverages the \emph{virtual-linkage model}, introduced by Williams and Khatib
	\cite{Williams_ICRA1993}, which provides a physical representation of the internal and CoM
	resultant forces with respect to reaction forces on the supporting surfaces
	\cite{Sentis_TransRobotics2010}.
	This virtual-linkage is a closed chain mechanism that represents the object being manipulated.
	This kinematic structure is composed by actuated prismatic joints connected by passive spherical
	joints, such that each spherical joint be connected to a each grasp point.
	Prismatic joints model internal forces (stresses throughout the object), while the spherical ones
	model internal moments \cite{Williams_ICRA1993}.
	
	\emph{Model-based control methods} can be used to enable fast, dexterous, and compliant motion of
	robots without sacrificing control accuracy \cite{Mistry_ICRA2010}, such that desired joint
	accelerations can be converted into joint torques using inverse dynamics for improved tracking
	performance \cite{Stephens_IROS2010}.
	However, implementing such techniques on floating base robots, e.g., humanoids and legged systems,
	is non-trivial due to comunications delays, under-actuation, dynamically changing constraints from
	the environment, potentially closed loop kinematics, structural compliance, mechanical bias, actuator
	stiction, and a variety of electromechanical phenomena \cite{Mistry_ICRA2010} \cite{Hopkins_ICRA2015}.
	In this context, Mistry et al. \cite{Mistry_ICRA2010} proposed in 2010 a relatively simple technique
	for fully-body model-based control of humanoid robots called \emph{Floating Body Inverse Dynamics}
	(FBID).
	Using an orthogonal decomposition of rigid-body dynamics, they were able to express the complete
	inverse dynamic equations of the robot independently of the contact forces, in order to cope with the
	under-actuation and the dynamically changing contact state inherent in these floating base systems
	\cite{Mistry_ICRA2010}.
	This approach was later extended to address optimal distribution of the contact forces by
	Righetti et al. \cite{Righetti_CLAWAR2010}
	
	Another model-based method known as the \emph{Dynamic Balance Force Control} (DBFC) was also introduced
	in 2010 by Stephens and Atkeson \cite{Stephens_IROS2010}, with the purpose of determining full body
	joint torques based on desired COM motion and contact forces for compliant humanoid robots
	\cite{Stephens_IROS2010}.
	This one uses contact force optimization for balancing the humanoid robot based on the CoM dynamics,
	by combining it with virtual task forces.
	For the mapping of the contact forces to the joint torques, the non-linear full multi-body dynamics are
	considered \cite{Ott_Humanoids2011}.
	However, the input is the desired contact forces.
	Contact forces are computed independent of the full robot model based on a simple COM dynamics model and
	external forces.
	Furthermore, because of the force-based nature of this controller, it can also be modified for the
	compensation of non-contact forces using VMC-like controls \cite{Stephens_IROS2010}.
	
	In 2011, Ott et al. \cite{Ott_Humanoids2011} proposed a balancing controller able to withstand external
	perturbations by distributing the required forces among predefined contact points via a	constrained
	optimization problem, in a similar way as proposed by Hyon et al. \cite{Hyon_TransRobotics2007}
	The approach reused a formulation coming from the field of robot grasping to create a controller that
	keeps both the position and orientation of the robot \cite{Ott_Humanoids2011}.
	This idea was later used by Henze et al. \cite{Henze_IROS2014} which proposed an approach to whole-body
	control for balancing and posture	stabilization of humanoid robots utilizing an optimization of contact
	forces in combination with Model Predictive Control \cite{Henze_IROS2014}.
	
	More recent approaches have explored the idea of simplifying the system dynamic equations by performing
	suitable projections onto the null-space of the contact forces, as proposed by Righetti et al.
	\cite{Righetti_Humanoids2011} based on recent results from analytical dynamics presented by Aghli
	\cite{Aghili_TransRobotics2005}.
	The geometric projector cancels the constraint forces from the system dynamics, removing any need
	of force measurements.
	Also, they are faster to compute than those depending on inertial quantities, as used by Sentis
	\cite{Sentis_PhDThesis2007} and Park \cite{Park_PhDThesis2006}, so resulting in control laws that were
	simpler and computationally more efficient (i.e., a total computation time below 1 ms)
	\cite{DelPrete_IROS2014} \cite{Nori_FrontRobAI2015}.
	Similar methods like Righetti et al. \cite{Righetti_ICRA2011}, and Mistry and Righetti
	\cite{Mistry_RSS2011} presented an in-between approach, extending the	Operational Space formulation
	of Khatib \cite{Khatib_RoboticsAuto1987} to underactuated constrained mechanical systems and proposing
	improved contact models \cite{DelPrete_IROS2014} \cite{Moro_IJHR2016}.
	However, the formulation of Mistry and Righetti	\cite{Mistry_RSS2011} was less efficient than the one
	of Righetti et al. \cite{Righetti_ICRA2011} because it used the inverse of the robot mass matrix
	\cite{DelPrete_IROS2014}.
	This one was later improved by Righetti et al. \cite{Righetti_IJRR2013}
	
	These approaches based on the elimination of the contact forces presented two major drawbacks.
	First, in general they cannot guarantee bounded contact forces.
	Second, every time the robot makes or breaks a contact, the discontinuity in the constraint set
	results in discontinuous control torques.
	These discontinuities may generate jerky movements or, even worse, make the robot slip and fall
	\cite{DelPrete_IROS2014}.
	
	Rather than finding an analytical solution of the control problem, an alternative approach suggested
	by De Lasa et al. \cite{DeLasa_IROS2009} is to use a Quadratic Programming (QP) solver.
	This allows to include inequality constraints into the problem formulation, which can model
	control tasks and physical constraints (e.g. joint limits, friction cones).
	For instance, Saab et al. \cite{Saab_ICRA2011} \cite{Saab_IROS2011} \cite{Saab_TransRobotics2013}
	used inequalities to account for the ZMP conditions on a walking humanoid.
	While this technique is appealing, solving a cascade of QPs with inequality constraints can be critical
	from a computational standpoint.
	Escande et al. \cite{Escande_IJRR2014} reached a computation time of 1 ms on an inverse kinematics
	problem at the price of seldom suboptimal solutions.
	However, they did not consider the inverse-dynamics, which has more than twice the number of variables
	\cite{DelPrete_IROS2014}.
	
	In 2012, Righetti and Schaal \cite{Righetti_Humanoids2012} proposed an inverse dynamics controller for
	a humanoid robot that exploited torque redundancy to minimize any combination of linear and quadratic
	costs in the contact forces and commands.
	Also, Nagasaka et al. \cite{Nagasaka_RobotSymp2012} proposed an elegant formulation of
	\emph{Model Preview Control} (MPC) using the CoM to stabilize a humanoid robot even when it
	operates keeping contact with multiple non-coplanar planes \cite{Audren_IROS2014}
	\cite{Nagasaka_RobotSymp2012}.
	This one was later implemented by Audren et al.	\cite{Audren_IROS2014} in the framework of 
	the multi-contact planner proposed by Escande et al. \cite{Escande_RobAutoSys2013}.
	
	In 2013, Del Prete \cite{DelPrete_PhDThesis2013} introduced a control framework for position and force
	control of constrained floating-base robots denominated as \emph{Task Space Inverse Dynamics}	(TSID)
	\cite{DelPrete_PhDThesis2013} \cite{DelPrete_RobAutoSys2015}.
	Under many aspects, it is similar to other control frameworks, such as the ones proposed by
	De Lasa and Hertzmann \cite{DeLasa_IROS2009}, Mistry and Righetti \cite{Mistry_RSS2011},
	Peters et al. \cite{Peters_AutoRobots2008}, Saab et al. \cite{Saab_ICRA2011}, and Sentis and Khatib
	\cite{Sentis_IJHR2005},	sharing with them most of its limitations \cite{DelPrete_PhDThesis2013}.
	One of the main flaws of the TSID method is that it does not take into account any of the uncertainties
	affecting the humanoid robots: poor torque tracking, sensor noises, delays and model uncertainties.
	As a consequence, the resulting control trajectories may be feasible for the ideal system,
	but not for	the real one.
	In this context, Del Prete and Mansard \cite{DelPrete_RSS2015} proposed to improve the robustness of
	TSID by modeling uncertainties in the joint torques as additive white random noise.
	This resulted in a stochastic optimization problem, in which they could maximize the probability
	to satisfy the inequality constraints (i.e. to be feasible) \cite{DelPrete_RSS2015}.
	
	Also in 2013, Moro \cite{Moro_Humanoids2013} \cite{Moro_IJHR2016} presented an attractor-based
	whole-body motion control	system, developed for	torque-control of floating-base robots performing
	simultaneous tasks.
	The \emph{attractors} are defined as atomic control modules that work in parallel to, and independently
	from the other attractors, generating joint torques that aim to modify the state of the robot so 
	that the error in a target condition is minimized.
	Each task is handled by an attractor, that is associated to a certain physical or derived measure
	(e.g., minimum effort, zero joint momentum, joint limits repeller, end effector position or force).
	One key idea behind of this novel implementation is the control of the effort of the robot without
	directly controlling it.
	The \emph{gravitational stiffness}, a physical quantity that is closely related to the effort,
	is controlled instead, as suggested by him \cite{Moro_Humanoids2015} in a previous work.
	Whenever the gravitational stiffness is maximized, the effort is minimized \cite{Moro_IJHR2016}.
	
	The implementation of hierarchical inverse dynamics controllers based on cascades of quadratic programs
	is not easy as model inaccuracies and real-time computation requirements can be problematic.
	However, in 2014 Herzog et al. \cite{Herzog_IROS2014} were able to evaluate these algorithms experimentally,
	showing that they can be used for feedback control of humanoid robots and that momentum-based
	balance control can be efficiently implemented on a real robot \cite{Herzog_IROS2014}.
	This work (improved later by them \cite{Herzog_Humanoids2015} \cite{Herzog_AutoRobots2016}) also proposed a
	method to simplify the optimization problem by factoring the dynamics equations	of the robot such that it was
	possible to	significantly reduce the computational time and achieve a 1 kHz control loop.
	Another example is the one of Feng et al. \cite{Feng_Humanoids2014} who achieved successful locomotion on
	the Atlas robot using PD feedback in conjunction with torque control.
	That is, these implementations of hierarchical inverse dynamics on purely torque controlled robots have
	required some form of joint-space position and/or velocity feedback to compensate for unmodeled dynamics
	\cite{Hopkins_ICRA2015}.
	
	In 2014, Del Prete et al. \cite{DelPrete_IROS2014} presented a method to control the motion and
	a subset of the contact forces of a floating-base robot, leading to an optimization with reduced
	computational complexity, comparable to inverse-dynamics based approaches \cite{DelPrete_IROS2014}.
	Concurrently, Lack et al. \cite{Lack_ICRA2014} presented a method for achieving planar multi-phase
	multi-contact robot walking using inspired control and optimization \cite{Lack_ICRA2014}.
	
	In 2015 the increasing interest of the research community in multi-contact control frameworks was
	notorious.
	Caron et al. \cite{Caron_RSS2015} proposed the use of a \emph{Gravito-Inertial Wrench Cone} (GIWC)
	as a general multi-contact stability cirterion; that is, a ``ZMP for non-coplanar contacts''.
	Farnioli et al. \cite{Farnioli_ICRA2015} presented a general framework for the quasi-static analysis
	of whole-body loco-manipulation problems.
	This one made use of the \emph{Fundamental Loco-Manipulation Matrix} (FLMM) and its canonical form
	to extract relevant information like the space of the controllable contact forces and the controllable
	displacements of the CoM \cite{Farnioli_ICRA2015}.
	Gams et al. \cite{Gams_Robotica2015} presented a method that allowed real-time motion imitation on
	the COMAN humanoid robot while maintaining stability, based on the prioritized task control proposed
	by Mistry et al. \cite{Mistry_IROS2007}.
	Koenemann et al. \cite{Koenemann_IROS2015} implemented a complete whole-body model-predictive controller
	based on the optimal-control solver MuJoCo developed by Todorov et al. \cite{Todorov_IROS2012}
	This one was implemented in real-time on the physical HRP-2 robot \cite{Koenemann_IROS2015}.
	Kudruss et al. \cite{Kudruss_Humanoids2015} presented a formulation of the reduced multi-contact CoM
	dynamics of a humanoid as an optimal control problem.
	This approach was then used to generate a whole-body motion of climbing stairs with support of a
	handrail, also for the physical HRP-2 robot \cite{Kudruss_Humanoids2015}.
	Liu and Padois \cite{LiuM_IROS2015} proposed a control mechanism to solve whole-body tasks under
	non-rigid contacts which worked online in a reactive way without requiring the knowledge of the
	rigidity of the environment \cite{LiuM_IROS2015}.
	Nori et al. \cite{Nori_FrontRobAI2015} detailed the implementation of the state-of-the-art whole
	body control algorithms on the humanoid robot iCub using a framework based on Del Prete et al.
	\cite{DelPrete_IROS2014}, which allows to control the contact forces, but with a computational
	complexity of the same order of inverse-dynamics-based methods \cite{Nori_FrontRobAI2015}.
	Sherikov et al. \cite{Sherikov_Humanoids2015} introduced a strict prioritization in contact force
	distribution, to reflect situations when an application of certain contact forces should be avoided
	as much as possible, for example, due to a fragility of the support.
	The controller is similar to a previous work also proposed by them \cite{Sherikov_Humanoids2014}
	which had considered a flat ground, improving it to deal with multiple non-coplanar contacts
	\cite{Sherikov_Humanoids2015}.
	
	Also in 2015, and inspired by Koolen et al. \cite{Koolen_Humanoids2013} and Herzog et al.
	\cite{Herzog_IROS2014}, Hopkins et al. \cite{Hopkins_ICRA2015} presented a compliant locomotion
	framework also using model-based whole-body control.
	In order to stabilize the centroidal dynamics during locomotion, they compute the linear momentum
	rate of change objectives using a	novel time-varying controller for the
	\emph{Divergent Component of Motion} (DCM), introduced by	Englsberber et al. \cite{Englsberger_IROS2013}
	and implemented for the first time on hardware on the THOR platform.
	This DCM is a 3D extension of the \emph{Capture Point} (CP) concept, introduced by Pratt et al.
	\cite{Pratt_Humanoids2006}, that simplifies locomotion planning and control on uneven terrain.
	Task-space objectives, including the desired momentum rate of change, were tracked using an efficient
	quadratic program formulation that computed optimal joint torque set-points given frictional contact
	constraints and joint position / torque limits.
	To improve the stability of the proposed inverse dynamics-based approach, joint velocity set-points
	were computed from the optimized joint accelerations and tracked using pre-transmission velocity
	estimates	\cite{Hopkins_ICRA2015}.
	This work was later extended by them \cite{Hopkins_IJHR2016} to include additional material related
	to the low-level control and state estimation implementation for the THOR humanoid as well as
	experimental results and analyssis related to locomotion on grass and gravel.
	
	In 2016, Koolen et al. \cite{Koolen_IJHR2016} presented a momentum-based torque control framework for
	floating-base robots and its application to the Atlas humanoid robot.
	This one was summarily introduced by them \cite{Koolen_Humanoids2013} in a previous work.
	It exploits the fact that the rate of change of whole-body \emph{centroidal momentum}
	is both affine in the joint acceleration vector and linear in the external wrenches applied to the
	robot \cite{Orin_IROS2008}.
	This fact is used to formulate a compact quadratic program, solved at every control time step, which
	reconciles motion tasks in the form of constraints on the joint acceleration vector with the available
	contacts between the robot and its environment in the form of unilateral contacts of force-limited
	grasps.
	Contact forces at unilateral contacts are constrained to remain within conservative polyhedral
	approximations of the expected friction cones \cite{Koolen_IJHR2016}.
	
	\emph{Momentum-Based Control} (MBC) has also gained popularity in recent years.
	Kajita et al. \cite{Kajita_IROS2003} were the first to propose a momentum based control scheme,
	called \emph{Resolve Momentum Control}.
	This scheme used unconstrained least squares to find joint velocity references to be tracked by
	a low level controller.
	The relationship between momentum and joint velocities was later studied in more detail by
	Orin and Goswami \cite{Orin_IROS2008}.
	Based on this work, they presented a momentum-based control framework for torque-controlled robots
	that has been used by Lee and Goswami	\cite{LeeS_IROS2010} \cite{LeeS_AutoRobots2012} and
	Orin et al. \cite{Orin_AutoRobots2013}
	These works then inspired the framework proposed by Hopkins et al. \cite{Hopkins_IJHR2016} and
	Koolen et al. \cite{Koolen_IJHR2016}.
	
	These frameworks have required specialized computational algorithms to compute the
	\emph{Centroidal Momentum Matrix} (CMM) and its derivative, which relate rates of change in
	centroidal momentum to joint rates and accelerations of the humanoid.
	However, these specialized algorithms are not always required.
	Wensing and Orin \cite{Wensing_IJHR2016} showed that since the dynamics of the centroidal momentum
	are embedded in the joint-space dynamics equation of motion, the CMM and terms involving its
	derivative can be computed from the joint-space mass matrix and Coriolis terms.
	This approach presented improvements in terms of its generality, compactness, and efficiency in
	comparison to previous specialized algorithms~\cite{Wensing_IJHR2016} and expanded previous
	works developed by them \cite{Wensing_ICRA2013} \cite{Wensing_IJHR2013}.
	
	A timeline of the works presented here is shown in Table~\ref{tab:Timeline}.
	
	\begin{table*}[t]
	%
	\caption{Timeline for works in multi-contact control.}
	\label{tab:Timeline}
	\scriptsize
	\centering
	%
	\begin{tabular}{|l|c|c|c|c|c|c|c|c|c|c|c|c|c|}
		\hline
		First Author	&
		2004	& 2005	& 2006	& 2007	& 2008	& 2009	& 2010	& 2011	& 2012	& 2013	& 2014	& 2015	& 2016	\\
		\hline
		Khatib O.			&
		\cite{Khatib_IJHR2004}	& & & & & & & & & & & & \\
		Park J.				&
		\cite{Park_ICRA2004}		& \cite{Park_ICRA2005}	& \cite{Park_ICRA2006} \cite{Park_PhDThesis2006}
		&	&	\cite{Park_Robotica2008}	& & & & & & & & \\
		Sentis L.			&
		\cite{Sentis_IJHR2004} & \cite{Sentis_ICRA2005} \cite{Sentis_IJHR2005}	& \cite{Sentis_ICRA2006}
		& \cite{Sentis_PhDThesis2007}	& & \cite{Sentis_IROS2009}
		& \cite{Sentis_MotionPlan2010} \cite{Sentis_TransRobotics2010}	&	& & & & & \\
		Hyon S.				&
		& & \cite{Hyon_Humanoids2006} \cite{Hyon_IROS2006}	& \cite{Hyon_ICRA2007} \cite{Hyon_TransRobotics2007}
		& & \cite{Hyon_ICRA2009} \cite{Hyon_TransRobotics2009}	& & & & & & & \\
		Mistry M.			&
		& & & \cite{Mistry_IROS2007} & & & \cite{Mistry_ICRA2010}	& \cite{Mistry_RSS2011}	& & & & & \\
		Orin D.				&
		& & & & \cite{Orin_IROS2008} & & & & & \cite{Orin_AutoRobots2013} & & & \\
		Peters J.			&
		& & & & \cite{Peters_AutoRobots2008} & & & & & & & & \\
		De Lasa M.		&
		& & & & & \cite{DeLasa_IROS2009}	& & & & & & & \\
		Stephens B.		&
		& & & & & & \cite{Stephens_IROS2010}	& & & & & & \\
		Lee S.				&
		& & & & & & \cite{LeeS_IROS2010}	& & \cite{LeeS_AutoRobots2012}	& & & & \\
		Righetti L.		&
		& & & & & & \cite{Righetti_CLAWAR2010} & \cite{Righetti_ICRA2011} \cite{Righetti_Humanoids2011}
		& \cite{Righetti_Humanoids2012}	& \cite{Righetti_IJRR2013}	& & & \\
		Ott C.				&
		& & & & & & & \cite{Ott_Humanoids2011}	& & & & & \\
		Saab L.				&
		& & & & & & & \cite{Saab_ICRA2011} \cite{Saab_IROS2011}	& & \cite{Saab_TransRobotics2013}	& & & \\
		Nagasaka K.		&
		& & & & & & & & \cite{Nagasaka_RobotSymp2012}	& & & & \\
		Koolen T.			&
		& & & & & & & & & \cite{Koolen_Humanoids2013} & & & \cite{Koolen_IJHR2016}	\\
		Escande A.		&
		& & & & & & & & & \cite{Escande_RobAutoSys2013} & & & \\
		Del Prete A.	&
		& & & & & & & & & \cite{DelPrete_PhDThesis2013}	& \cite{DelPrete_IROS2014}
		& \cite{DelPrete_RSS2015} \cite{DelPrete_RobAutoSys2015}	& \\
		Moro F.					&
		& & & & & & & & & \cite{Moro_Humanoids2013} & & \cite{Moro_Humanoids2015} & \cite{Moro_IJHR2016}	\\
		Wensing P.			&
		& & & & & & & & & \cite{Wensing_ICRA2013} \cite{Wensing_IJHR2013}	& & & \cite{Wensing_IJHR2016}	\\
		Audren H.			&
		& & & & & & & & & & \cite{Audren_IROS2014}	& & \\
		Herzog A.			&
		& & & & & & & & & & \cite{Herzog_IROS2014} & \cite{Herzog_Humanoids2015}
		& \cite{Herzog_AutoRobots2016}	\\
		Feng S.				&
		& & & & & & & & & & \cite{Feng_Humanoids2014}	& & \\
		Lack J.				&
		& & & & & & & & & & \cite{Lack_ICRA2014}	& & \\
		Caron S.			&
		& & & & & & & & & & & \cite{Caron_RSS2015}	& \\
		Farnioli E.		&
		& & & & & & & & & & & \cite{Farnioli_ICRA2015}	& \\
		Gams A.				&
		& & & & & & & & & & & \cite{Gams_Robotica2015}	& \\
		Koenemann J.	&
		& & & & & & & & & & & \cite{Koenemann_IROS2015}	& \\
		Kudruss M.		&
		& & & & & & & & & & & \cite{Kudruss_Humanoids2015}	& \\
		Liu M.				&
		& & & & & & & & & & & \cite{LiuM_IROS2015}	& \\
		Nori F.				&
		& & & & & & & & & & & \cite{Nori_FrontRobAI2015}	& \\
		Sherikov A.		&
		& & & & & & & & & & & \cite{Sherikov_Humanoids2015}	& \\
		Hopkins M.			&
		& & & & & & & & & & & \cite{Hopkins_ICRA2015}	& \cite{Hopkins_IJHR2016}	\\
		\hline
	\end{tabular}
	%
\end{table*}
	
	\section{Remarks}
	\label{sec:remarks}
	
	Many control frameworks have then been formulated in the last decade, some of them clearly
	representative, e.g. Inverse-Kinematics Force Control (IKFC) \cite{Fujimoto_ICRA1996},
	Virtual Model Control (VMC) \cite{Pratt_IJRR2001}, Operational Space Based Control
	(OSBC) \cite{Sentis_MotionPlan2010}, Passivity-Based Balance Control (PBBC)
	\cite{Hyon_TransRobotics2007}, Floating Body Inverse Dynamics (FBID) \cite{Mistry_ICRA2010},
	Dynamic Balance Force Control (DBFC) \cite{Stephens_IROS2010}, Model Preview Control (MPC)
	\cite{Nagasaka_RobotSymp2012}, Task Space Inverse Dynamics (TSID) \cite{DelPrete_PhDThesis2013},
	Momentum-Based Control (MBC) \cite{Hopkins_IJHR2016}, among others.
	
	Each of these frameworks present particular features \cite{DelPrete_PhDThesis2013}, such as:
	specification of inequality constraints \cite{Mansard_TransRobotics2009} \cite{Saab_ICRA2011},
	control of underactuated systems \cite{DeLasa_IROS2009} \cite{Mistry_RSS2011}, low computational
	cost \cite{Escande_ICRA2010} \cite{Mansard_ICRA2012}, etc., but in general they pursue one
	common objective: to create complex behaviors, for which humanoid robots need to simultaneously
	accomplish multiple control objectives.
	For instance, locomotion, manipulation, balance, and posture stance need to be simultaneously
	controlled \cite{Sentis_PhDThesis2007}.
	
	It is possible to identify two different approaches to multi-task management among the described
	frameworks \cite{DelPrete_PhDThesis2013}:
	%
	\begin{enumerate}
		\item The \emph{error minimization} approach.
					This one minimizes the error of each task, under the constraint of not conflicting
					with higher priority tasks.
					Unfortunately, it suffers from algorithmic singularities, namely singularities that
					are due to conflicts between tasks.
					
		\item The \emph{null-space projection} approach.
					This one is somewhat simpler as it only projects tasks into the null-space of the
					constraints.
					However, it has a drawback, as it does not ensure that a	task is achieved,
					even if it is not in conflict with any task with higher priority.
		
	\end{enumerate}
		
	Despite the drawbacks of the null-space projection approach sometimes it has been preferred
	over the error minimization approach \cite{DelPrete_PhDThesis2013}, as it was done by
	Khatib	\cite{Khatib_IJRR1995}, Chiaverini \cite{Chiaverini_RoboticsAuto1997},
	Peters et al. \cite{Peters_AutoRobots2008}, or Mistry and Righetti \cite{Mistry_RSS2011}. 
	This is because
	%
	\begin{inparaenum}
		\item it is older and hence more popular in the robotics community,
		\item it is simpler to understand and to implement, and
		\item it does not suffer from algorithmic singularities.
	\end{inparaenum}
	%
	However, it is not optimal \cite{DelPrete_PhDThesis2013}.
		
	On the other hand, if the robot has to perform two or more tasks with different priorities,
	then the error minimization approach has to be taken to get optimal results.
	Many frameworks for multi-task control that follow the principles of this approach have been
	presented.
	Some examples are the works of Siciliano and Slotine \cite{Siciliano_AdvRobotics1991},
	Jeong and Chang \cite{Jeong_IROS2009} or Saab et al. \cite{Saab_ICRA2011}.
	
	Even the work of Khatib	\cite{Khatib_IJRR1995}, which in its original form used the null-space
	projection approach, has been extended later to the error minimization approach in the works
	of Khatib et al. \cite{Khatib_IJHR2004}, or Sentis and Khatib \cite{Sentis_ICRA2006}, as it
	was interpreted by Del Prete \cite{DelPrete_PhDThesis2013}.
		
	In a general walking robot, the number of constraints is larger than that of a finger system.
	Therefore, it may be difficult to implement this method in real-time without reducing the
	problem size \cite{Kwon_IntellRobotSys1998}.
	This reduction of the problem size represents an approach that can improve the performance
	of this method, as proposed by Chen et al. \cite{Chen_MIRC1999}.
	
	The use of Quadratic Programming (QP) in online control algorithms has recently seen a sudden
	rise in popularity,	at least in part due to the availability of fast and reliable QP solvers
	and more powerful CPUs \cite{Koolen_IJHR2016}.
	An early example of such QP-based control schemes is the work of Kudoh et al. \cite{Kudoh_IROS2002}.
	More recent approaches include the work of Stephens and Atkeson \cite{Stephens_IROS2010},
	Saab et al. \cite{Saab_TransRobotics2013}, Kuindersma et al. \cite{Kuindersma_ICRA2014},
	Herzog et al. \cite{Herzog_IROS2014}, and Feng et al. \cite{Feng_Humanoids2014}
	
	Several researchers have proposed convex optimization techniques to solve the inverse dynamics and
	whole body control problem subject to multi-contact constraints, e.g. De Lasa and Hertzmann
	\cite{DeLasa_IROS2009}, Lee and Goswami \cite{LeeS_IROS2010}, Koolen et al. \cite{Koolen_Humanoids2013},
	Wensing and Orin \cite{Wensing_ICRA2013}, Feng et al. \cite{Feng_Humanoids2014}, Saab et al.
	\cite{Saab_TransRobotics2013} and Kuindersma et al. \cite{Kuindersma_ICRA2014}
	These approaches compute joint torque set-points that minimize tracking errors for multiple motion
	tasks including momentum rates of change, end-effector accelerations, and joint accelerations relating
	to whole-body motions.
	In general, these formulations can serve as the basis for any locomotion, manipulation, or generic
	multi-contact behavior \cite{Hopkins_ICRA2015}.
	With respect to this multi-contact behavior many works in the literature have only focused on
	controlling forces exerted at the	end-effectors, namely the hands for manipulation tasks and
	the feet for walking.
	Restricting contacts to end-effectors is a quite strong assumption, since
	%
	\begin{inparaenum}
		\item uncertainties in the environment may result in unplanned contacts at other body parts,
					and
		\item contacts at other body parts may be necessary to perform certain tasks
					\cite{DelPrete_PhDThesis2013}.
	\end{inparaenum}
	
	Some works do not make any assumption about the contact location, trying to control the joint
	torques	rather than the contact forces.
	This approach ensures bounded contact forces, but do not allow a real contact force control
	\cite{DelPrete_PhDThesis2013}.
	
	Various theoretical frameworks for modeling and control of robots that are subject to multiple
	contacts have been proposed (e.g. Park and Khatib \cite{Park_ICRA2006}, Sentis et al.
	\cite{Sentis_IROS2009} and Righetti et al. \cite{Righetti_ICRA2011}), but they have not
	gone beyond end-effector contacts when tested	on real platforms \cite{DelPrete_PhDThesis2013}.

	One of the first schemes for model-based torque control was due to Khatib
	\cite{Khatib_RoboticsAuto1987}, later extended and applied to humanoid robots.
	Such a framework allowed the specification of (a hierarchy of) multiple motion tasks,
	and could handle the case of multiple non-coplanar ground contacts.
	Then, joint torques were essentially found using unconstrained least squares (pseudo-inverse) techniques
	\cite{DelPrete_PhDThesis2013}.
	Examples of this type of scheme include the work of Hyon et al. \cite{Hyon_TransRobotics2007},
	Righetti et al. \cite{Righetti_CLAWAR2010} and Mistry et al. \cite{Mistry_ICRA2010}
	
	A consequence of this unconstrained optimization is that limitations due to unilateral ground contact
	and friction cannot be taken into account directly \cite{Koolen_IJHR2016}.
	Also, the control laws based on the assumption of no motion along the constrained directions can
	generate constraint forces that are not physically feasible.
	Righetti et al.	\cite{Righetti_Humanoids2011} tried to solve this issue by using the constraint
	redundancy to minimize a quadratic cost in the constraint forces, but still it couldn't guarantee
	their physical consistency \cite{DelPrete_PhDThesis2013}.
	
	A safer approach was to include inequality constraints into the control problem, as done by Saab et al.
	\cite{Saab_ICRA2011} \cite{Saab_IROS2011} or later by Righetti and Schaal	\cite{Righetti_Humanoids2012},
	so that it allowed for unilateral contact constraints \cite{DelPrete_PhDThesis2013}.
	This consideration naturally led to the use of constrained optimization	techniques.
	But, even though the friction cone constraints related to Coulomb friction were second-order cone
	constraints, these second-order cones were often approximated using linear constraints, reducing the
	problem of finding joint torques given motion tasks to a quadratic program (QP) \cite{Koolen_IJHR2016}.
	This is computationally more expensive than a solution based on pseudo-inverses; however, it is
	claimed that the computation time of the controller can be still less than 1 ms, and so perfectly
	suitable for fast torque control loop \cite{DelPrete_PhDThesis2013}.
	
	One limitation of works like the one of Del Prete \cite{DelPrete_PhDThesis2013} is that their control
	framework didn't consider joint limits and motor torque limits.
	These limits can be easily included in the control problem as inequality constraints, using the QP
	solver to compute the control torques \cite{DelPrete_PhDThesis2013}, as it is done by Hopkins et al.
	\cite{Hopkins_IJHR2016} and \cite{Koolen_IJHR2016}.
	
	\section{Future Work}
	\label{sec:future_work}
	
	Some of the recent works on multi-task whole-body multi-contact control have recognized weaknesses
	that have to be overcome by future research efforts.
	Some relevant ones are mentioned in the following:
	%
	\begin{itemize}
					
		\item \emph{Contact behavior}.
					It is advisable for future work to include the implementation of extreme contact behaviors,
					such as those that would exploit point and edge contacts to maneuver with the environment,
					instead of relying on planar surfaces.
					Also, it would be interesting to analyze contact singularities \cite{Sentis_TransRobotics2010}.
					
		\item \emph{Wrench measurement}.
					It is required that the humanoid robots have the capability of simultaneously measuring forces
					and	torques (i.e., wrenches) at any possible contact location.
					This is not possible with conventional torque-controlled manipulators and requires whole-body
					distributed force and tactil sensing \cite{Nori_FrontRobAI2015}.
					Also, it is worth to consider that, for future robot skin, composed of thousands or millions
					of sensors, it would be extremely challenging and time consuming to perform a full spatial
					calibration \cite{Calandra_Humanoids2015}.
		
		\item \emph{State estimation}.
					Future research efforts will have to focus on improved humanoid state estimation, as the
					accuracy of the floating base odometry is believed to be a limiting factor in the current
					implementations	\cite{Hopkins_IJHR2016}.
					
		\item \emph{Robustness and adaptation}.
					Accounting for uncertainties could lead to major improvements in robotics if applied at all
					levels (i.e. planning, control, estimation and identification).
					While robust Model Predictive Control (MPC) is already an active research field, applications
					of robust optimization in robotics are seldom \cite{DelPrete_LAAS2015}.
					Also, if whole-body control approaches become increasingly robust, new opportunities could
					arise for optimization-based locomotion planners that incorporate novel task-space and
					joint-space cost terms and additional constraints \cite{Hopkins_IJHR2016}.
					
		\item \emph{Singularities}.
					Whole-body algorithms that compute joint accelerations based on desired Cartesian accelerations
					require the inversion of a matrix, which can be singular.
					Singularities such as this have been a major issue in robotics and one of the main reasons why
					so many humanoid robots walk with significantly bent knees even though real humans walk with
					fairly straight	knees.
					There have been so many different methods for compensating for singularities, such as using
					damped least squares, but there is still no satisfying solution \cite{Koolen_IJHR2016}.
					
		\item \emph{Failure handling}.
					It is currently interesting to have appropriate methods to handle uncertainty and failure,
					specifically regarding the position and orientation of the assumed contact points, and also
					including the application of adaptive control methods for dynamic step recovery
					\cite{Lengagne_IJRR2013} \cite{Hopkins_IJHR2016}.
					Also, future applications will require automatic generation of stable behaviors.
					In addition, controllers should improve performance and efficiency of behaviors through
					experience \cite{Stephens_PhDThesis2011}.
					
		\item \emph{Complex motions}.
					It is a general goal to gradually increase the complexity of the generated motion:
					to have the robot making dynamic motion while making and breaking contacts with the rigid
					(or non-rigid)	environment, e.g. walking with additional support of the hands
					\cite{Koenemann_IROS2015}.
					It is to be noted that, despite the various studies on angular momentum in humanoid motions,
					the issue of how to set the desired angular momentum for more complex motions has not been
					fully explored \cite{LeeS_AutoRobots2012}.
					
		\item \emph{Computational cost}.
					Several solutions have only been tested in simulation due to the high computational cost of the
					algorithms involved, as well as due to several bottlenecks (e.g. in planning), which makes it
					difficult to allow a real-time execution on the real robot \cite{Kudruss_Humanoids2015}.
					
		\item \emph{Hardware and system integration}.
					Not only will humanoid robot controllers need to improve, but the hardware and system integration
					need to be improved to allow robots capable of real world application \cite{Stephens_PhDThesis2011}.
	
	\end{itemize}
	

	
	

	
	

	

	

	
	

	

	
	
	
	\bibliographystyle{plain}
	\bibliography{MultiContactSurvey}
	
\end{document}
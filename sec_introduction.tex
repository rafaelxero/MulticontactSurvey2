\section{Introduction}
	\label{sec:introduction}

	\subsection{Motivation: position control vs. force control}
		\label{sub:motivation}
		
		Robotic control is most frequently accomplished with a \emph{stiff position control} system
		\cite{Khatib_ICRA2008}, which tries to follow a desired position trajectory considering external
		forces as disturbances \cite{DelPrete_PhDThesis2013}.
		This typical controller is realized at the joint level using PD control with high gains.
		Thus, each joint is treated	independently	and, since the controller cannot account for the dynamics
		of the system, coupled dynamic effects between joints are ignored and treated as a disturbances.
		This approach has been well suited in factory automation, where joint control is a dominant
		paradigm in \emph{industrial robots} \cite{Khatib_ICRA2008}.
		
		Similar to industrial robots, \emph{humanoid robots} have been traditionally based on joint
		position control, which in combination with a stiff and precisely manufactured drive train
		leads to a high achievable precision \cite{Englsberger_Humanoids2015}.
		However, this type of control generally exhibits \emph{high impedance}, and the speed at which
		it will comply to an unknown force is restricted \cite{Stephens_IROS2010}.
		Furthermore, it is also severely limited in performing advanced robotic tasks, especially in
		unstructured dynamic complex environments \cite{Khatib_ICRA2008}, where it is important for the
		robot not only to adapt	quickly to sudden environmental changes or disturbances, especially to
		unknown external forces	\cite{Hyon_TransRobotics2007}, but also to interact closely with people.
		On the other hand, robots with \emph{low impedance} joints can comply faster.
		This is useful, but also makes balance control more important and more difficult \cite{Stephens_IROS2010}.
		That insufficiency is an important disadvantage for position-based controllers
		in terms of force	interaction \cite{Hyon_TransRobotics2007}.
		
		In a \emph{position-based compliance control} the robot adapts compliantly to external forces by using a
		force sensor.
		Such methods however, can deal only with forces via \emph{virtual compliance}; for that reason,
		time delays are, in principle, unavoidable.
		Moreover, the controller requires force sensors at every expected contact point.
		Without them, the robot cannot adapt to unknown external forces.
		Although it is possible to install force sensors over the whole body,
		this imparts a heavy computational burden because of the number of contact Jacobians.
		Also, it is difficult to assign appropriate weights.
		The best way to treat unknown disturbances and external forces is to transform desired contact
		(interaction) forces into joint torques directly, even though controlling these joint torques
		precisely is technically challenging \cite{Hyon_TransRobotics2007}.
		
		One alternative approach for addressing this problem(s)	is to use
		\emph{joint torque sensing and control}	\cite{Khatib_ICRA2008} \cite{Ott_Humanoids2011}.
		This kind of control has recently attracted increased attention, mainly due	to the fact that
		it is possible to interact robustly with the environment, as well as exhibit a safe
		and compliant behavior while interacting with humans (probably the most crucial issue) and
		in case of self-collisions \cite{Englsberger_Humanoids2015}.
		
	\subsection{Humanoids: from balance to multi-contact force control}
		\label{sub:humanoid_robots}
		
		Research and development of humanoid robots started in 1973 with the development of WABOT-1
		in Waseda	University, and the first successful walking humanoid system presented by Honda in
		1995 \cite{Park_PhDThesis2006}.
		Since then, the research had mainly focused on the hardware design and the realization of some
		basic	motions such as walking and balancing \cite{Harada_ICRA2003}.	
		
		Humanoid balance depends critically on controlling the linear and angular momentum of the	system,
		quantities that can be directly controlled by contact forces \cite{Stephens_IROS2010}.
		In general, a \emph{contact} can be seen as a continuum of infinitesimal forces acting on
		the surface of a rigid body, being the effect of contact forces represented with an
		equivalent wrench, composed by a three-dimensional force and a three-dimensional torque.
		\cite{Nori_FrontRobAI2015}.
		
		Given a robot with stiff joint position control and a known environment, the most common approach
		to balance is to generate a stable trajectory of the COM and then track it using inverse kinematics
		(IK) \cite{Stephens_IROS2010}.
		These trajectories can be modified online to produce whole body balance in the presence of small
		disturbances \cite{Sugihara_IROS2002}.
		For environments with small uncertainty or small disturbances, the inverse kinematics can be modified
		to directly control the contact forces using force feedback, a method known as Inverse Kinematics
		Force Control (IKFC) \cite{Fujimoto_ICRA1996}.
		
		Since the kinematical structure of a humanoid robot is similar to that of a human, a humanoid
		robot	is expected to work in the same environment as the human.
		Then, to accomplish the required tasks under such an environment, it should be considered that
		a	humanoid robot manipulates an object by means of a cooperation between arms and legs
		\cite{Harada_ICRA2003}; that is, due to multi-contact interactions.
		
		One of the first works considering the arm / leg coordination of a humanoid robot is the one of
		Inoue et al. \cite{Inoue_ICRA2000}, which determined the posture of a humanoid robot taking the
		manipulability of the arms into consideration \cite{Harada_IROS2003}.
		In this work, an impedance control was implemented to make the arm tips always follow their desired
		position, while considering	an external force	being applied \cite{Inoue_ICRA2000}.
		
		Forces were also considered within a force-balanced biped robot (not for multi-contact control)
		by Pratt et al. \cite{Pratt_IJRR2001}
		This work introduces the Virtual Model Control (VMC), a motion control framework that uses
		simulations of virtual components to create virtual forces generated when the virtual components
		interact with a robotic system.
		This is the simplest method that only uses a kinematic model \cite{Stephens_IROS2010}.
		
		The first research considering a pushing manipulation by a humanoid robot
		(and the influence of	the corresponding external forces in a multi-contact scenario)
		was reported by Harada et al. \cite{Harada_ICRA2003}
		These authors considered an extension to the Zero Moment Point (ZMP) dynamic-evaluation
		criterion, defined as the Generalized Zero Moment Point (GZMP), in order to enable generalized
		multicontact-locomotion behaviors.
		This extension was later used by them \cite{Harada_IROS2003} for arm / leg coordination.
		
		A formulation of multi-contact compliant motion control was, in our knowledge, first addressed
		by Park et al. \cite{Park_ICRA2004}
		This one was based on the contact model described by Featherstone et al. \cite{Featherstone_ICRA1999}
		together with the operational space formalism introduced by Khatib \cite{Khatib_RoboticsAuto1987},
		which made it possible to use force control to attain compliant contacts.
		Although the target of this work was not a humanoid robot, it addressed the case in which one
		link of a manipulator	was making contact with the environment in more than one point, providing
		the basics for future humanoid-related research in multi-contact compliant motion control.
		
	\subsection{Challenges in real-time force control of humanoid robots}
		\label{sub:challenges}
		
		The real-time multi-contact force control of a humanoid robot involves the optimal distribution of an
		undetermined force system subjected to physical constraints \cite{Chen_MIRC1999}
		\cite{Nahon_TransRoboticsAuto1992}.
		\emph{Force distribution} is the inverse dynamics problem for multiple-chain systems of links (arms,
		legs, or fingers) supporting a load, body or object, in which the motion is completely specified and
		the internal forces / torques to effect	this motion are to be determined \cite{Orin_AdvRobotics1989}.
		However, the solution to these \emph{inverse dynamics} equations of a multi-legged system is not unique
		(it is an ill-posedness problem),	but it can be chosen in an optimal manner by introducing an
		objective function \cite{Hyon_TransRobotics2007} \cite{Chen_MIRC1999}.
		
		Physical interaction with the environment definitely represents a crucial issue:
		while we can estimate a dynamical model of the robot, it is much harder to retrieve a model of the
		\emph{physical constraints} (the \emph{contact dynamics}), for which simplified models are usually
		used \cite{DelPrete_PhDThesis2013}.
		However, even though some of these ones can be simplified and represented by equality constraints,
		many more can only be represented mathematically as inequalities due to the nature of the
		contacts involved, i.e., normal contact forces between the feet and the ground cannot be negative,
		and, as supporting feet should not slip, the magnitude of the tangential force at each foot cannot
		exceed the maximum force of	static friction.
		In addition, the torque of each joint must lie within the allowed range.
		Therefore, force distribution requires to find a solution for a nonlinear programming problem
		with inequality constraints, a problem that should be solved in real-time, although it is difficult
		to do so due to the computational cost \cite{Chen_MIRC1999}.
		This is because the results of this force distribution are used, after an appropriate force-to-torque
		transformation, to provide torque set points for the actuators \cite{Nahon_TransRoboticsAuto1992}.